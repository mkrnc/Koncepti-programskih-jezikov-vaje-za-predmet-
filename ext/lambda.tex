\chapter{Lambda ra\v cun}


\begin{ex}
Napi\v si izraz v lambda ra\v cunu, ki za dana parametra x in y izra\v cuna povpre\v cno vrednost x in y.

Apliciraj prej definiran lambda izraz na konkretnih parametrih in zapi\v si redukcijo izraza do vrednosti.
\end{ex}




\begin{ex}
Uporabi alfa konverzijo in beta redukcijo za ovrednotenje naslednjega stavka zapisanega z $\lambda$-ra\v cunom
$$(\lambda a.\lambda b.a (a b)) (\lambda a.a + 1)\,1.$$

a) Izra\v cunaj vrednost izraza.

b) Kaj naredi funkcija?
\end{ex}




\begin{ex}
Evaluiraj naslednje lambda izraze do vrednosti.

1) $( \lambda f.f( \lambda x.x))( \lambda x.x) $

2) $( \lambda x. \lambda y.x) z w$

\end{ex}



\begin{ex}
Dane imamo naslednje standardne kombinatorje lambda ra\v cuna.
\begin{align*}
I& \equiv \lambda x.x;\\
K& \equiv \lambda x.\lambda y.x;\\
K^*& \equiv \lambda x.\lambda y.y;\\
S& \equiv \lambda x. \lambda y.\lambda z.xz(yz)
\end{align*}
Poenostavi naslednje izraze:
\begin{align*}
M &\equiv (\lambda x.\lambda y.\lambda z.zyx)aa(\lambda p.\lambda q.q);\\
M &\equiv (\lambda y.\lambda z.zy)((\lambda x.xxx)(\lambda x.xxx))(\lambda w.I);\\
M &\equiv SKSKSK
\end{align*}
\end{ex}



\begin{ex}
V naslednjih $\lambda$-izrazih prika\v zi vse oklepaje.  
\begin{itemize}
\item $(\lambda x.xa)ax$ 
\item $(\lambda z.zxz)(\lambda y.yx)z$ 
\end{itemize}
\end{ex}




\begin{ex}
Poi\v s\v ci vse proste (nevezane) spremenljivke v naslednjih $\lambda$-izrazih. 
\begin{itemize}
\item $(\lambda b.xba)xb$ 
\item $\lambda x.zy\lambda y.yx $
\end{itemize}
\end{ex}



\begin{ex}
Napi\v si naslednji izraz z \v cim manj oklepajev. 

 \begin{itemize}
 \item $((xy)(\lambda y.(\lambda z.(z(\lambda y.(xy)))x)y)) $
 \end{itemize}
\end{ex} 



