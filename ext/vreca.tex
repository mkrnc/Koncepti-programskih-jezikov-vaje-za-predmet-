%te naloge so že vnešene v glavni dokument

\chapter[Nekategorizirano]{Nekategorizirane naloge (stari  izpiti)}

\section{Polja}

\begin{ex}
  Slika ra\v cunalni\v ske naprave je definirana z naslednjim tipom

  \begin{verbatim}
  type slika = int array array,
  \end{verbatim}

  kjer so pike, ki sestavljajo sliko predstaljene z 0 ali 1. Pika je
  osvetljena, \v ce ima vrednost 1.

  Napi\v si funkcijo

  \begin{verbatim}
  xor : slika -> slika -> slika,
  \end{verbatim}

  ki dve slike kombinira tako, da naredi operacijo \texttt{xor} nad
  istole\v znimi pikami.

  Predpostavimo, da so vhodne slike istih dimenzij.


\end{ex} 
\begin{ex}
  Urejevalnik besedil ima tekst predstavljen z dvo-dimenzionalno
  matriko znakov tipa \texttt{tekst}.

  \begin{verbatim}
  type tekst = char array array;;
  \end{verbatim}

  Napi\v si funkcijo \texttt{poisci : tekst -> string -> int*int
    list}, ki vrne seznam koordinat kjer se za\v cne v tekstu iskani niz
  podan kot drugi parameter. 


\end{ex} 
\begin{ex}
  Urejevalnik besedil ima tekst predstavljen z dvo-dimenzionalno
  matriko znakov tipa \texttt{tekst}.

  \begin{verbatim}
  type tekst = char array array;;
  \end{verbatim}

  Napi\v si funkcijo \texttt{poisci\_vertikalno : tekst -> string ->
    int*int list}, ki vrne seznam koordinat kjer se v tekstu za\v cne
  vertikalni niz znakov podan kot drugi parameter.


\end{ex} 
\begin{ex}
  Slika neke ra\v cunalni\v ske naprave je definirana z naslednjim
  tipom

  \begin{verbatim}
  type slika = int array array,
  \end{verbatim}

  kjer so pike, ki sestavljajo sliko predstavljene z 0 ali 1. Pika je
  osvetljena, \v ce ima vrednost 1.

  \begin{enumerate}[label=(\roman*)]
  \item Napi\v si funkcijo \texttt{zasukaj90 : slika -> slika}, ki
    zasu\v ce sliko za 90 stopinj v smeri urinega kazalca.

  \item Napi\v si funkcijo \texttt{zasukaj90xn : slika -> int ->
      slika}, ki zasu\v ce sliko za kot \texttt{n*90°} v smeri urinega
    kazalca, kjer je \texttt{n} drugi parameter funkcije.
  \end{enumerate}

  Pazi na dimenzije slik!


\end{ex} 
\begin{ex}
  Definiraj tip \texttt{slika} s katerim predstavimo sliko sestavljeno
  iz 100x100 to\v ck. Vsaka to\v cka je predstavljana z intenziteto in
  barvo---obe vrednosti predstavimo s celim \v stevilom.

  Predpostavljamo, da imamo \v ze napisano funkcijo \texttt{pika :
    int*int -> bool}, ki pove ali je na dani koordinati pika. Barva
  pike ni pomembna.

  Nekje na sliki je narisan krog z radijem 5. Napi\v si funkcijo
  \texttt{poisci : slika -> (int*int)}, ki poi\v s\v ce sredi\v s\v ce
  kroga.


\end{ex} 
\begin{ex}
  Urejevalnik besedil ima tekst predstavljen z dvo-dimenzionalno
  matriko znakov.

  a) Definiraj tip tekst, ki predstavlja dvo-dimenzionalno matriko
  znakov velikosti 100x1000 (1000 vrstic po 100 znakov)..

  b) Napi\v si funkcijo

\begin{verbatim}
zamenjaj_navpicno : tekst -> string -> string -> tekst, 
\end{verbatim}

  ki poi\v s\v ce vse pojavitve niza (2. parameter) vertikalno v
  tekstu urejevalnika (1. parameter) in jih zamenja z drugim nizom
  (3. parameter). Predpostavimo, da sta niza enako dolga.


\end{ex} 

\section{Seznami}

\begin{ex}
  \label{dnk-naloga}
  Dano imamo sekvenco DNK v obliki seznama znakov tipa \texttt{dnk list},
  kjer je tip \texttt{dnk} definiran kot

  \begin{verbatim}
  type dnk = A | C | T | G
  \end{verbatim}
 
  Na primer, niz znakov "ACAAGT" je predstavljen s seznamom 
  \texttt{[A;\-C;A;A;G;T]}. 

  Napi\v si funkcijo \texttt{najpodniz : dnk list -> int*int}, ki
  poi\v s\v ce najdalj\v si podniz istih znakov tipa \texttt{dnk} ter
  izpi\v se pozicijo prvega znaka in dol\v zino niza.


\end{ex} 
\begin{ex}
  Dano imamo sekvenco DNK v obliki seznama znakov tipa \texttt{dnk list},
  kjer je tip \texttt{dnk} definiran kot v Nalogi \ref{dnk-naloga}.

  Napi\v si funkcijo \texttt{prestej : dnk list -> dnk*int list}, ki
  se\v steje enake zaporedne znake v sekvenci in za vsako skupino
  kreira par, ki vsebuje \texttt{dnk} znak in \v stevilo pojavitev
  znaka.

  \emph{Primer}: 
  \begin{verbatim}
  # prestej [A;C;A;A;G;C;C];;
  val - : dnk*int list = [(A,1);(C,1);(A,2);(G,1);(C,2)]
  \end{verbatim}


\end{ex} 
\begin{ex}
  Napi\v si funkcijo \texttt{zamenjaj : int list -> int list}, ki v
  enem prehodu poi\v s\v ce v seznamu celih \v stevil vse zaporedne
  pare \texttt{x::y}, ki niso urejeni po nara\v s\v cajo\v cem vrstnem
  redu (\texttt{x<=y}) in jih obrne.

  Z uporabo funkcije \texttt{zamenjaj} implementiraj sortiranje seznama.


\end{ex} 
\begin{ex}
  Relacijo \texttt{R} predstavimo s seznamom parov celih \v
  stevil. Napi\v si funkcijo

  \begin{verbatim}
  razsiri : int*int list -> int list -> int list,
  \end{verbatim}

  kjer je prvi parameter seznam parov \texttt{r} in drugi parameter
  seznam celih \v stevil \texttt{s}. Rezultat funkcije
  \texttt{razsiri} naj bo seznam, ki vsebuje vsa \v stevila \texttt{x}
  za katera velja \texttt{(e,x)}$\in$\texttt{R} za nek element e
  seznama \texttt{s}.


\end{ex} 
\begin{ex}
  Napi\v si funkcijo

  \begin{verbatim}
  zdruzi : int list -> int list -> int list,
  \end{verbatim}

  ki zdru\v zi urejena seznama celih \v stevil tako, da je rezultat
  urejen in hkrati izlo\v ci ve\v ckratne ponovite elementov.


  \begin{sol}
  \begin{verbatim}
  let rec zdruzi list1 list2 = match (list1, list2) with
  | ([], []) -> []
  | (a, []) -> a
  | ([], b) -> b
  | (a::b, c::d) -> 
       if (a < c ) then a::(zdruzi b (c::d))
       else if (c<a) then c::(zdruzi (a::b) d)
            else a::(zdruzi b d)
  \end{verbatim}
  \end{sol}

\end{ex} 

\section{Tipi}

\begin{ex}
   Izrazi zelo enostavnega jezika \texttt{TP} so sestavljeni iz
   vrednosti med katerimi so postavljene operaciji \texttt{TIMES} ali
   \texttt{PLUS}. Obe operaciji sta levo asociativne vendar ima
   operacija \texttt{TIMES} vi\v sjo prioriteto od
   \texttt{PLUS}. Primer izraza je:

   \begin{verbatim}
   1 PLUS 2 TIMES 3 TIMES 4,
   \end{verbatim}

   kar ustreza aritmeti\v cnem izrazu $1 + ((2 * 3) * 4)$. Izraze jezika
   TP lahko definiramo z naslednjimi tipi:
 
   \begin{verbatim}
   # type operation = PLUS | TIMES;; 
   type operation = PLUS | TIMES 
   # type element = Val of int | Op of operation;; 
   type element = Val of int | Op of operation
   # type expr = list element;;
   type expr = list element
   \end{verbatim}

   \begin{enumerate}[label=(\roman*)]
   \item Napi\v si funkcijo \texttt{check : expr -> bool}, ki preveri
     ali je izraz pravilno napisan.

   \item Napi\v si funkcijo \texttt{calc : expr -> int}, ki izra\v
     cuna vrednost izraza.
   \end{enumerate}


\end{ex}

\subsection{Parametrizirani tipi}

\begin{ex}
  Definiraj parametriziran tip \texttt{'a seznam} z uporabo zapisov!
  Zapis naj ima dve komponenti: vrednost tipa \texttt{'a} in kazalec na
  naslednji zapis v seznamu oz. prazen seznam.

  Napi\v si funkcijo \texttt{dolzina : 'a seznam -> int}, ki
  pre\v steje \v stevilo vrednosti v seznamu.


\end{ex} 
\begin{ex}
  Definiraj parametri\v cni tip \texttt{slika}, ki predstavlja
  dvo-dimenzionalno ra\v cunalni\v sko sliko z neznanim tipom
  elementov (pik) polja, ki ga ozna\v cimo z \texttt{'a}.

  Napi\v si funkcijo vi\v sjega reda

  \begin{verbatim}
  zdruzi : 'a slika -> 'a slika -> ('a->'a->'a) -> 'a slika,
  \end{verbatim}

  ki ima prva dva parametra sliki \texttt{s1} in \texttt{s2} ter
  tretji parameter funkcijo \texttt{f : 'a->'a->'a}, ki zdru\v zi dve
  piki v eno piko. Funkcija \texttt{zdruzi} zdru\v zi istole\v zne
  pike iz slik \texttt{s1} in \texttt{s2} z uporabo funkcije
  \texttt{f}.


\end{ex} 

\subsection{Polimorfi\v cne funkcije}

\begin{ex}
  Napi\v si polimorfi\v cno funkcijo
  \begin{verbatim}
  ostevilci : 'a list -> (int * 'a) list, 
  \end{verbatim}
  ki o\v stevil\v ci elemente od $1$ do $n$, \v ce je n dol\v zina
  vhodnega seznama. Funkcija ostevilci naj vrne seznam parov, kjer je
  prva komponenta indeks in druga komponenta orginalna vrednost iz
  seznama.

\end{ex} 

\section{Funkcije vi\v sjega reda}

\begin{ex} 
  Napi\v si funkcijo vi\v sjega reda

  \begin{verbatim}
  skrci : (int * int -> int) -> list int -> int,
  \end{verbatim}

  katere prvi parameter \texttt{f : int*int -> int} je funkcija, ki
  zdru\v zi dve vrednost iz seznama v en samo vrednost tipa
  \texttt{int}. Drugi parameter \texttt{skrci} je seznam celih \v
  stevil.

  Naj bo funkcija \texttt{f} prvi parameter in seznam \texttt{l = [i1;
    i2; ...; in] } drugi parameter. Pomen funkcije \texttt{skrci}
  lahko zapi\v semo na naslednji na\v cin.

  \begin{verbatim}
  skrci f l = f i1 (f i2 f(...(f in-1 in)))
  \end{verbatim}

  Predpostavi, da ima seznam \texttt{l} vsaj dva elementa!


\end{ex}

\section{Rekurzivni tipi}

\subsection{Seznami}

\begin{ex}
  Seznam definiramo z zapisom, ki ima dve komponenti: vrednost tipa
  \texttt{'a} in kazalec na naslednji element seznama tipa \texttt{'a
    seznam}, ki je bodisi prazen ali pa ima \v se vsaj en element.

  \begin{verbatim}
  type 'a element = { 
     mutable vrednost:'a; 
     mutable naslednji:'a seznam 
  }
  and 'a seznam = Prazen | Element of 'a element ;;
  \end{verbatim}

  Napi\v si funkcijo \texttt{podvoji : 'a seznam -> 'a seznam}, ki
  podvoji vse vrednosti v seznamu.


\end{ex} 
\begin{ex}
  Podatkovna struktura \texttt{seznam} je definirana na naslednji na\v
  cin.

  \begin{verbatim}
  type 'a element = { 
     mutable vrednost: 'a; 
     mutable naslednji: 'a seznam 
  }
  and 'a seznam = Prazen | Element of 'a element ;;
  \end{verbatim}

  Napi\v si funkcijo \texttt{zdruzi : 'a seznam -> 'a seznam -> 'a
    seznam}, ki ima za argumenta dva seznama tipa \texttt{'a seznam}.

  \begin{enumerate}[label=(\roman*)]
  \item Funkcija naj zdru\v zi seznama tako, da pove\v ze konec prvega
    seznama z za\v cetkom drugega ter kot rezultat vrne za\v cetek prvega
    seznama.

  \item Funkcija naj zdru\v zi seznama tako, da naredi kopijo obeh
    seznamov v novem seznamu, ki vsebuje na novo konstruirane
    elemente.
  \end{enumerate}


\end{ex} 
\begin{ex}
  Dan je tip \texttt{2seznam} s katerim je predstavljen dvojno povezan
  seznam.

  \begin{verbatim}
  type 2seznam = {
     value: int;
     mutable next: 2seznam;1
     mutable prev: 2seznam
  } ;;
  \end{verbatim}

  Cela \v stevila hranimo po nara\v s\v cajo\v cem vrstnem
  redu. Napi\v si funkcijo \texttt{odaj : 2seznam -> int -> 2seznam}
  ki doda \v stevilo (2. parameter) na pravo mesto v seznam.


\end{ex} 

\subsection{Drevesa}

\begin{ex}
  Dan je tip \texttt{grm}, ki je definiran na slede\v c na\v cin:

\begin{verbatim}
type 'a grm = 
    Nic 
  | Ena of 'a * 'a grm 
  | Dva of 'a grm * 'a * 'a grm;;
\end{verbatim}

  Napi\v si funkcijo \texttt{dolzinevej : 'a grm -> unit}, ki izpi\v
  se dol\v zine vej grma po principu levo-v-globino.


\end{ex} 
\begin{ex} 
  Podan je tip \texttt{'a drevo} s katerim je predstavljeno binarno
  drevo.

  \begin{verbatim}
  type 'a drevo = { 
      mutable levo:'a bin_drevo; 
      mutable vozlisce:'a; 
     mutable desno:'a bin_drevo 
  } 
  and 'a bin_drevo = Prazen | Vozlisce of 'a drevo ;;
  \end{verbatim}

  Napi\v si funkcijo, ki izpi\v se vsa vozli\v s\v ca tretjega nivoja
  drevesa, ki obstajajo v drevesu.


\end{ex} 
\begin{ex}
  Binarno drevo je definirano s tipom
  \begin{verbatim}
  type bdrevo = List of int | Drevo of bdrevo * int * bdrevo,
  \end{verbatim}
  ki ima vrednosti v listih definirane kot \texttt{List}$(v)$, kjer je
  $v$ celo \v stevilo.  Notranja vozli\v s\v ca so definirana kot
  trojica \texttt{Drevo}$(l, v, d)$, kjer sta $l$ in $d$ levo in desno
  poddrevo, $v$ pa je vrednost, ki je za\v cetno $0$ za vsa notranja
  vozli\v s\v ca.

  \begin{enumerate}[label=(\roman*)]
  \item Napi\v si funkcijo
    \begin{verbatim}
    prestej : bdrevo -> int, 
    \end{verbatim}
    ki vrne vsoto vrednosti podanih v listih vhodnega drevesa.

  \item Napi\v si funkcijo
    \begin{verbatim}
    oznaci : bdrevo -> bdrevo,
    \end{verbatim}
    ki drugo komponento vseh notranjih vozli\v s\v c izhodnega drevesa
    zamenja z vsoto vseh listov poddrevesa.
  \end{enumerate}


\end{ex} 

\section{Razredi}


\begin{ex}
  V svetu eno-dimenzionalnih robotov se roboti pomikajo po premici
  levo in desno. Premica ima izhodi\v s\v ce, ki ima vrednost
  0. Roboti se lahko pomikajo za korak levo ali korak desno. 

  Imamo dve vrsti robotov, ki se premikajo po podro\v cju [-100..100]:

  \begin{enumerate}
  \item Robot se premakne za eno enoto v izbrano smer. Ko pride do
    meje se obrne. To vrsto robota predstavimo z razredom
    \texttt{robot1}.

  \item Robot se premika naklju\v cno za eno enoto desno ali levo. To
    vrsto robota predstavimo z razredom \texttt{robot2}.
  \end{enumerate}

  \begin{enumerate}[label=(\roman*)]
  \item Definiraj robote z razredi povezanimi v hierarhijo
    dedovanja. Vrh hierarhije dedovanja naj bo razred \texttt{robot}.

  \item Definiraj za vse robote metodo \texttt{premik}, ki naredi
    naslednji premik robota.

  \item Definiraj razred \texttt{robot3} kot podrazred razreda
    \texttt{robot1}. Z uporabo prekrivanja definiraj premik tega
    robota kot dvakraten premik robota iz razreda \texttt{robot1}.
  \end{enumerate}


\end{ex} 
\begin{ex}
  V dvo-dimenzionalnem svetu robotov imamo dve vrsti robotov:
  x-robota, ki se premika samo po x-osi in y-robota, ki se premika
  samo po y-osi. Svet ima dimenzije -10..10 po x-osi in -10..10 po
  y-osi.

  Premik robota po x-osi implementiramo tako, da pri\v stejemo
  x-koordinati 1 oz. -1, odvisno od tega ali se robot premika desno
  ali levo. Ko robot pride do roba sveta zamenjamo premik iz 1 v -1
  oz. obratno.

  Premikanje robota po y-osi je definirano enako kot v primeru
  premikanja po x-osi le da so osi zamenjane in se robot premika
  navzgor in navzdol.

  \begin{enumerate}
  \item Definiraj razrede s katerimi predstavimo x-robota in
    y-robota. Definiraj skupen koren hierarhije razredov z uporabo
    abstraktnega razreda.

  \item V abstraktnem razredu robot definiraj virtualno metodo
    \texttt{premik}. Implementiraj metodo \texttt{premik} v
    okviru obeh konkretnih razredov.
  \end{enumerate}


\end{ex} 
\begin{ex}

  \begin{enumerate}
  \item Definiraj razred \texttt{zbirka}, ki hrani zbirko celih \v
    stevil urejeno po vrstnem redu dolo\v cenim z uporabo funkcij za
    dodajanje elementov. Implementiraj naslednje metode:

    \begin{verbatim}
    dodaj_zacetek : int -> unit
    brisi_zacetek : int 
    dodaj_konec : int -> unit
    brisi_konec : int  
    \end{verbatim}

  \item Razreda \texttt{vrsta} (FIFO) in \texttt{sklad} (FILO)
    implementiraj kot specializaciji razreda
    \texttt{zbirka}. Definiraj ustrezne metode razredov \texttt{vrsta}
    (enqueue in dequeue) in \texttt{sklad} (push in pop).
  \end{enumerate}

  \emph{Nasvet}: Za implementacijo razreda \texttt{zbirka} uporabi \v
  cim bolj enostavno podatkovno strukturo!


\end{ex} 
\begin{ex}
  Naprave so sestavljene iz manj\v sih naprav, ki so spet lahko
  sestavljene iz manj\v sih naprav, itd. Vsaka naprava ima svoje ime,
  tip in te\v zo.
  
  \begin{enumerate}[label=(\Alph*)]
  \item Definiraj razred \texttt{naprava}.

  \item Definiraj metodo \texttt{listi : naprava list} razreda
    \texttt{naprava}, ki v seznamu vrne liste drevesa vsebovanosti
    naprav za dano napravo (objekt).
  \end{enumerate}


\end{ex} 

\subsection{Parametrizirani razredi}

\begin{ex}
  Definiraj binarno drevo z objekti, ki predstavljajo vozli\v s\v
  ca drevesa. Definiraj parametriziran razred \texttt{vozlisce}
  katerega parameter tip \texttt{'a} naj predstavlja tip vrednosti
  vozli\v s\v ca.

  Pod-drevesi danega vozli\v s\v ca definiraj kot opcijske vrednosti s
  \v cimer se izognemo definiciji praznega pod-drevesa. Opcijske
  vrednosti definiramo z uporabo tipa \texttt{'a option}.

  \begin{verbatim}
  type `a option = Some of `a | None
  \end{verbatim}

  Napi\v si metodo, ki vstavi novo vozli\v s\v ce v skrajno levo vejo
  drevesa.

 
\end{ex} 



-----------------------------------------------------------------------


\begin{ex}
  V programskem jeziku Ocaml imamo definirano podatkovno strukturo
  seznam na naslednji na\v cin:

  type 'a element = { 
     mutable vrednost:'a; 
     mutable naslednji:'a seznam 
  } 
  and 'a seznam = Prazen | Element of 'a element ;; 

  Napi\v si funkcijo zdruzi : 'a seznam -> 'b seznam -> ('a * 'b)
  seznam, ki zdru\v zi elemente dveh seznamov v en sam seznam tako, da
  prepi\v se istole\v zne elemente v pare. \v ce ne obstaja istole\v
  zni par potem naj ima komponenta vrednost () : unit.

 
\end{ex} 
\begin{ex}
  Imamo naprave, ki so sestavljene iz manj\v sih naprav, ki so spet
  lahko sestavljene iz manj\v sih naprav, itd. Vsaka naprava ima svoje
  ime in tip. Naprave definiramo z naslednjim razredom naprava.

\begin{verbatim}
# class naprava (im:string) (tp:string) (ln:naprava list) = 
object 
    val ime = im 
    val tip = tp 
    val mutable komponente = ln 
end;; 
class naprava : 
  string -> 
  string -> 
  naprava list -> 
  object 
    val ime : string 
    val mutable komponente : naprava list 
    val tip : string 
  end 
\end{verbatim}

  Napi\v si metodo \texttt{izpisi\_liste}, ki izpi\v se imena in tipe
  vseh listov—naprav, ki nimajo nobene komponente.



\end{ex} 
\begin{ex}
  Napi\v si funkcijo cikli : int -> int -> int, ki za klic cikli m n
  generira seznam n ciklov \v stevil od 0 do m-1.

\noindent\/Primer:
\begin{verbatim}
# cikli 3 4;; 
- : int list = [0; 1; 2; 0; 1; 2; 0; 1; 2; 0; 1; 2]
\end{verbatim}

\begin{sol}
\begin{verbatim}
let rec stevila n m = let x=m-1 in
if n > x then []
else n :: stevila (n+1) m;;

let rec cikli m n = match n with
| 0 -> []
| _ -> (stevila 0 m)@cikli m (n-1)
\end{verbatim}
\end{sol}

\end{ex} 
\begin{ex}
  Napi\v si polimorfi\v cno funkcijo rapp :
  ('a->'a)->int->('a->'a). Klic rapp f n applicira funkcijo f n-krat
  na danem parametru:

  rapp f n = (function x -> f (...(f x)...))     

\noindent\/Primer:            
\begin{verbatim}
# let f x = x+3;; 
val f : int -> int = <fun> 
# (rapp f 3) 1;; 
- : int = 10 

\end{verbatim}

Namig: Uporabi\v s lahko funkcijo compose:

\begin{verbatim}
# let compose f g x = f (g x) ;; 
val compose : ('a -> 'b) -> ('c -> 'a) -> 'c -> 'b = <fun> 
\end{verbatim}

\end{ex} 
\begin{ex}
  Definiraj uporabni\v ski tip 'a btree, ki predstavlja binarno drevo
  katerega vozli\v s\v ca vsebujejo vrednost tipa 'a.

  Napi\v si funkcijo izomorfni : 'a btree -> 'a btree -> bool, ki
  preveri \v ce imata podani drevesi enako strukturo ne glede na
  vrednosti v vozli\v s\v cih.


\end{ex} 
\begin{ex}
  a) Definiraj razred matrika, ki predstavlja mxn matrike realnih \v
  stevil. Razred naj se inicializira s parametroma m in n.

  b) Definiraj naslednje metode razreda Matrika:

  get : int -> int -> float
  set : int -> int -> float -> unit
  mul : matrika -> unit

  Vse metode spreminjajo matriko predstavljeno z objektom, ki izvr\v
  si metodo.  Na primer, metoda mul pomno\v zi dano matriko (objekt) z
  matriko, ki je podana kot parameter. Rezultat je shranjen v matriko
  (objekt), ki izvr\v si metodo.

  c) Definiraj podrazred kmatrika, ki predstavlja kvadratno matriko
  nxn.


\end{ex} 
\begin{ex}
  Dana je funkcija fib3, ki je definirana na slede\v c na\v cin:

  fib3(n) = 1, za n=1,2,3
  fib3(n) = fib3(n-1)+fib3(n-2)+fib3(n-3), za n>3.

  Napi\v si funkcijo fib3list : int -> int list, ki generira seznam
  vrednosti funkcije fib3 od 1 do n, kjer je n>0 parameter funkcije.

\noindent\/Primer:
\begin{verbatim}
# fib3 6;;
_ : int list = [1; 1; 1; 3; 5; 9]
\end{verbatim}

%\begin{sol}
%\begin{verbatim}
%let rec fib3 x = match x with
%| 0 -> []
%| _ -> match fib3(x-1) with
%| [] -> [1]
%|	[_] -> [1;1]
%| [;] -> [1;1;1]
%|	a::b::c::_ as smallfib -> (a+b+c)::smallfib;;
%
%let rec obrni sez = match sez with
%| [] -> []
%| g::r -> (obrni r)@[g];;
%
%obrni (fib3 6);;
%
%note: not optimal. feel free to edit. I'm not planing on doing HM with this
%\end{verbatim}
%\end{sol}

\end{ex} 
\begin{ex}
  Dan je seznam parov, ki vsebujejo klju\v c tipa 'k in vrednost tipa
  'v. Napi\v si funkcijo

  filter ('v->bool) -> 'k*'v list -> 'k list,

  ki iz seznama parov dolo\v cenim z 2. parametrom izbere tiste klu\v
  ce za katere vrne funkcija ('v->bool) dolo\v cena s 1. parameterom
  vrednost true. Rezultat funcije filter je seznam tak\v snih klju\v
  cev.

\noindent\/Primer:
\begin{verbatim}
# filter (function x -> x=0) [(1,0);(2,1);(3,0);(4,1)];; 
- : int list = [1; 3] 

\end{verbatim}

\end{ex} 
\begin{ex}
  Dan je tip 'a struc definiran na naslednji na\v cin.

  type 'a struc = 
      Elm of 'a 
    | Pair of 'a struc * 'a struc 
    | Triple of 'a struc * 'a struc * 'a struc

  Napi\v si polimorfi\v cno funkcijo

  smap : ('a->'b) -> ('a struc) -> ('b struc),

  ki aplicira funkcijo f dolo\v ceno s 1. parametrom na vseh
  komponentah Elm x, ki so del 2. parametra. Rezultat naj bo struktura
  tipa 'b struc.

\noindent\/Primer:            
\begin{verbatim}
# smap (function x -> x+1) 
       (Triple (Pair (Elm 1, Elm 2),Elm 3, Elm 4));; 
- : int struc = Triple (Pair (Elm 2, Elm 3), Elm 4, Elm 5) 
\end{verbatim}

\end{ex} 
\begin{ex}
  Binarno drevo je definirano s tipom itree, ki predstavlja binarno
  drevo katerega vozli\v s\v ca vsebujejo vrednost tipa int.

  type itree = Nil | Node of itree*int*itree;;

  Napi\v si funkcijo sumsub : itree -> itree, ki iz vhodnega drevesa
  konstruira novo drevo, ki namesto originalnih vrednosti v voli\v s\v
  cih vsebuje vsoto vrednosti vozli\v s\v c levega in desnega
  pod-drevesa ter vrednosti danega vozli\v s\v ca.

\noindent\/Primer:
\begin{verbatim}
# sumsub Node(Node(Nil,3,Nil),5,Node(Nil,2,Node(Nil,1,Nil)));;
_ : itree = Node(Node(Nil,3,Nil),11,Node(Nil,3,Node(Nil,1,Nil)))
\end{verbatim}


\end{ex} 
\begin{ex}
  Dano imamo tabelo funkcij definirano z naslednjim tipom

  type 'a ftab = ('a->'a) array

  Napi\v si funkcijo

  tapply : int*'a list -> 'a ftab -> 'a list.

  Naj ima funkcija tapply l t parametra l, ki je seznam parov oblike
  (i,a) in parameter t, ki je tabela funkcij. Funkcija tapply pretvori
  l v seznam vrednosti, ki jih dobimo z aplikacijo funkcij t.(i) na a.


\end{ex} 
\begin{ex}
  Definiraj parametri\v cni \texttt{('k,'v) ppolje}, ki predstavlja
  polje parov 'k*'v. Prva komponenta para je klju\v c tipa 'k in druga
  vrednost tipa 'v.

  \noindent\/Predpostavljaj:

  a) Imamo dano funkcijo enako : 'k->'k->bool, ki definira enakost
  klju\v cev, in funkcijo zdruzi : 'v->'v->'v, ki zdru\v zi dve
  vrednosti tipa 'v v eno samo.

  b) Polja tipa ('k,'v) ppolje so vedno sortirana po vrednosti klju\v
  ca 'k!

  Napi\v si funkcijo:

  stik ('k,'v) ppolje -> ('k,'v) ppolje -> ('k,'v) ppolje,

  ki naredi stik polj tako, da s funkcijo zdruzi zdru\v zi vrednosti
  vseh parov, ki se ujemajo v klju\v cu. Rezultat naj bo torej seznam
  parov tipa ('k,'v).


\end{ex}
\begin{ex}
  Dani so tip 'a izraz, ki predstavja aritmeti\v cne izraze nad
  vrednostmi tipa 'a

  type operacija = PLUS|MINUS;;
  type 'a izraz = 'a | 'a izraz * operacija * 'a izraz;;

  ter funkciji plus : 'a->'a->'a in minus : 'a->'a->'a, ki izra\v
  cunata vsoto in razliko vrednosti tipa 'a.

  Napi\v si funkcijo izracun : 'a izraz -> 'a, ki izra\v cuna vrednost
  izraza tipa 'a izraz.

\noindent\/Primer:
\begin{verbatim}
# izracun (2,plus,(3,minus,1));;
- : int = 4
\end{verbatim} 

\end{ex} \begin{ex}
Definiraj hierarhijo razredov za geometrijske objekte to\v cko, krog in kvadrat. Poskusi v \v cim ve\v cji meri uporabiti gradnike objektno-usmerjenega modela za u\v cinkovito predstavitev hierarhije. 

a) Definiraj razrede tocka, krog in kvadrat. Uporabi abstraktni razred za vrh hierarhije razredov.

b) Za vse geometrijske objekte definiraj metodo premakni, ki premakne dan geometrijski objekt. Uporabi dedovanje in prekrivanje metod!

c) Opi\v si potek inicializacije objekta ob kreiranju novega objekta.

\end{ex} \begin{ex}
Napi\v si funkcijo v Ocaml, ki za dani seznam celih \v stevil se\v steje soda in liha \v stevila, ter vrne par, ki ima na prvem mestu vsoto lihih \v stevil, na drugem pa vsoto sodih \v stevil.


\end{ex} \begin{ex}
Definiraj parametri\v cni tip 'v ppolje, ki predstavlja polje (array!) parov string*'v. Prva komponenta para je klju\v c tipa string in druga vrednost tipa 'v. 

Predpostavljaj, da je polje tipa 'v ppolje sortirano po vrednosti klju\v ca tipa string!  

Napi\v si funkcijo

stik 'v ppolje -> 'v ppolje -> 'v ppolje, 

ki naredi stik dveh polj tako, da rezultat vsebuje pare obeh polj urejenih po klju\v cu tipa string. 


Primer:

\begin{verbatim}
# stik [|("ab",10),("de",9) |] [| ("bc",8),("cd",12)|];;
- : int ppolje = [|("ab",10),("bc",8),("cd",12),("de",9)|]

\end{verbatim}

\end{ex} \begin{ex}
Dan je tip bdrevo, ki je definiran na naslednji na\v cin

type bdrevo = Leaf of int | Node of bdrevo * int * bdrevo;;

Napi\v si funkcijo bapply : bdrevo -> (int -> int) -> bdrevo, ki aplicira funkcijo (int -> int) na vseh vozli\v s\v cih drevesa.  


\end{ex} \begin{ex}
Hi\v sa je sestavljena iz N nadstropij. Vsako nadstropje ima M sob. V vsaki sobi imamo termometer. Temperaturo hi\v se dolo\v cimo tako, da izra\v cunamo povpre\v cje meritev temperature v vseh sobah. 

Predpostavimo, da imamo dano funkcijo temperatura : int -> int -> int, ki za dano nadstropje in \v stevilko sobe vrne vrednost temperature v stopinjah. 

a) Definiraj razrede s katerimi predstavimo: sobe, nadstropja in celotno hi\v so.
b) Vsak razred mora imeti metodo odcitajTemperaturo, ki vrne temperaturo objekta in  postavi trenutno vrednost temperature za dan objekt.
c) Razred hi\v sa mora izra\v cunati temperaturo hi\v se po zgoraj opisanem postopku.


\end{ex} \begin{ex}
a) V naslednjih $\lambda$-izrazih prika\v zi vse oklepaje.  
\begin{itemize}
\item $(\lambda x.xa)ax$ 
\item $(\lambda z.zxz)(\lambda y.yx)z$ 
\end{itemize}
    
b) Poi\v s\v ci vse proste (nevezane) spremenljivke v naslednjih $\lambda$-izrazih. 
\begin{itemize}
\item $(\lambda b.xba)xb$ 
\item $\lambda x.zy\lambda y.yx $
\end{itemize}

c) Napi\v si naslednji izraz z \v cim manj oklepajev. 

 \begin{itemize}
 \item $((xy)(\lambda y.(\lambda z.(z(\lambda y.(xy)))x)y)) $
 \end{itemize}


\end{ex} \begin{ex}
a) Z uporabo unije definiraj tip 'a element, ki predstavi elemente naslednjih oblik:

    1. elemente tipa 'a ali
    2. sezname elementov tipa 'a element.

    Primer elementa: 
\begin{verbatim}
    val a : int element = L [E 1; E 2; L [E 3; E 4]]
\end{verbatim}

b) Napi\v si funkcijo print : 'a element -> unit, ki izpi\v se elemente po pravilu                          
    najprej-levo-v-globino.



\end{ex} \begin{ex}
Izrazi enostavnega jezika z imenom TP vsebujejo cela \v stevila ter operaciji PLUS in TIMES. Predpostavimo, da imata operaciji isto prioriteto. Naslednji izraz
\begin{verbatim}
1 PLUS 2 TIMES 3 TIMES 4,
\end{verbatim}
ustreza aritmeti\v cnem izrazu ((1 + 2) * 3) * 4). Izrazi jezika TP so definirani z naslednjimi tipi. 
 
\begin{verbatim}
# type operation = PLUS | TIMES;; 
type operation = PLUS | TIMES 
# type element = Val of int | Op of operation;; 
type element = Val of int | Op of operation
# type expr = list element;;
type expr = list element

\end{verbatim}
Napi\v si funkcijo calc : expr -> int, ki izra\v cuna vrednost danega izraza.  


\end{ex} \begin{ex}
Predpostavi, da je definiran razred Array, ki vsebuje polje elementov tipa 'a. 

\begin{verbatim}
class ['a] Array (ini: 'a) =
object 
     method size : int
     method set : int -> 'a -> unit
     method get : int -> 'a
end
\end{verbatim}
Definiraj podrazred ArrayM, ki za dan indeks tipa int lahko vsebuje ve\v c kot eno vrednost tipa 'a.  Definiraj naslednje metode razreda. 

get : int -> 'a list             (vrni elemente z indeksom i) 
set : int -> 'a -> unit    (dodaj element tipa 'a k elementom z indeksom i)
del : int -> 'a -> unit   (izbri\v si el. tipa 'a iz mno\v zice elementov z indeksom i)

\v cim bolje uporabi metode razreda Array. 


\end{ex} \begin{ex}
a) Definiraj parametri\v cni tip slovar, ki vsebuje seznam parov. Prvi element parov naj vsebuje klju\v c tipa int in drugi element vsebuje vrednost poljubnega tipa. 

b) Napi\v si funkcijo 

preberi : slovar -> int -> 'a, 

kjer 'a predstavlja tip druge komponente para v slovarju. Funkcija preberi vrne za dan klju\v c (2. parameter) iz slovarja (1. parameter) pripadajo\v co vrednost tipa 'a.


\end{ex} \begin{ex}
Matrika je predstavljena s tipom int array array. Napi\v si funkcijo 

podniz : int array array -> int array -> bool,

ki preveri ali se niz celih \v stevil (2. parameter) pojavi v matriki (1. parameter) na kateri izmed diagonalnih \v crt, ki poteka iz to\v ck na levi in spodnji strani matrike navzgor proti desni in zgornji strani matrike.


\end{ex} \begin{ex}
Dan je tip drevo, ki je definiran na slede\v c na\v cin:

\begin{verbatim}
# type 'a drevo =
  List
| Veja of  'a * 'a drevo
| Rogovila of 'a drevo * 'a * 'a drevo;;

\end{verbatim}
Napi\v si funkcijo dolzinevej : 'a drevo -> int list , ki izpi\v se seznam globin listov drevesa po principu levo-v-globino. Pri tem velja, da je top vozli\v s\v ce na nivoju 0.

Primer uporabe:

\begin{verbatim}
# dolzinevej (Rogovila (Veja (4,List),6,
              Rogovila (List,7,Veja (8,List))));;
−: int list = [2;2;3]

\end{verbatim}

\end{ex} \begin{ex}
Dan je razred Sklad, ki realizira obi\v cajen sklad celih \v stevil

\begin{verbatim}
class Sklad =
object 
     method size : int
     method push : int -> unit
     method pop : int
end

\end{verbatim}
Definiraj razred Vrsta, ki implementira vrsto z dvema skladoma. Prvi sklad predstavlja za\v cetek vrste in drugi sklad predstavlja konec vrste. Razred Vrsta naj vsebuje operaciji: 

\begin{verbatim}
enqueue : int -> unit, 
dequeue : int. 

\end{verbatim}
V primeru, da je drugi sklad prazen in uporabimo operacijo dequeue, potem najprej obrnemo prvi sklad in ga shranimo kot drugi sklad.



\end{ex} \begin{ex}
Drevo je definirano z naslednjim tipom. 

\begin{verbatim}
type bindrevo = List of int | Drevo of bindrevo * bindrevo ;; 
\end{verbatim}

Napi\v si funkcijo izpis : bindrevo -> int -> unit, ki izpi\v se vse liste katerih vrednost je ve\v cja od drugega parametra. 


\end{ex} \begin{ex}
Definiraj parametri\v cni tip key\_value, ki predstavlja zapis z dvema komponentama, prva komponenta je klju\v c tipa 'a in druga komponenta je vrednost tipa 'b. 

Na osnovi tipa key\_value definiraj parametri\v cni tip slovar, ki je implementiran s poljem! 

Dano imamo polimorfi\v cno funkcijo equal : 'a -> 'a -> bool, ki vrne true v primeru, da je prvi parameter enak drugemu in false sicer. 

Napi\v si funkcijo duplikati : slovar -> slovar, ki iz slovarja odstrani vse duplikate. 


\end{ex} \begin{ex}
Seznam imamo definiran na nasledni na\v cin. 
\begin{verbatim}
# type 'a rnode = { mutable cont:'a; mutable next:'a rlist } 
   and 'a rlist = Nil | Elm of 'a rnode;; 
\end{verbatim}
Napi\v si funkcijop filter : 'a rlist -> ('a -> 'a -> bool) -> 'a rlist, ki vrne elemente seznama (1. parameter) za katere funkcija podana z 2. parametrom vrne vrednost true.

Funkcijo filter napi\v si tako, da ohrani\v s kopije elementov, oz. tako, da se ne kreira nov seznam!


\end{ex} \begin{ex}
Dan je razred sklad, ki realizira parametriziran sklad celih \v stevil
\begin{verbatim}
class ['a] Sklad =
object 
     method size : 'a
     method push : 'a -> unit
     method pop : 'a
end
\end{verbatim}
Definiraj razred Kalkulator, ki bo osnova za implementacijo enostavnega kalkulatorja, ki deluje z uporabo obrnjene poljske notacije. Kalkulator deluje na naslednji na\v cin.  

\v stevilo, ki ga vnesemo da na vrh delovnega sklada. Operacije plus, minus, mnozi in deli vzamejo zadnja dva operanda iz sklada, opravijo operacijo in vrnejo rezultat na sklad. 

Kako bi naredili splo\v sen kalkulator, ki lahko uporablja poljubno predstavitev \v stevil?


\end{ex} \begin{ex}
Napi\v si funkcijo vi\v sjega reda 

\begin{verbatim}
foldx : 'a list -> 'b -> ('a -> 'b -> 'b) list -> 'b,
\end{verbatim}

ki nad seznamom, ki je podan s prvim argumentom, in za\v cetno vrednostjo podano z drugim argumentom, aplicira funkcije iz seznama podanega kot tretji argument na slede\v c na\v cin. 

Naj bo prvi argument enak [a1,a2,...,an], drugi argument b, in tretji argument enak [f1,f2,...,fn]. Rezultat dobimo na slede\v c na\v cin: 

f1 a1  (f2 a2 ... (fn  an b) ...)


\end{ex} \begin{ex}
Definiraj funkcijo 

zmesaj : 'a array -> (int*int) array -> 'a array, 

ki kot prvi parameter sprejme polje vrednosti tipa 'a, kot drugi parameter pa polje parov s katerimi so definirane zamenjave elementov. Funkcija naj vrne zakodirano polje. 

Polje parov vsebuje pare indeksov s katerimi je definirana zamenjava dveh elementov vhodnega polja. 


\end{ex} \begin{ex}
Dano je drevo, ki vsebuje dve vrsti elementov in je definirano z naslednjo podatkovno strukturo: 

\begin{verbatim}
type ('a, 'b) tree = 
  Nil
| Nodea of 'a * ('a, 'b) tree list 
| Nodeb of 'b * ('a, 'b) tree list;; 

\end{verbatim}
Napi\v si funkcijo 

\begin{verbatim}
razcepi: ('a,'b) drevo -> 'a list * 'b list, 
\end{verbatim}

ki prepi\v se vse elemente Nodea v prvi seznam, ki postane prvi element vrnjenega para, in vse elemente Nodeb v drugi seznam, ki postane drugi element vrnjenega para.


\end{ex} \begin{ex}
Podan je modul Stack, ki realizira sklad elementov tipa 'a.. 

\begin{verbatim}
module type Stack = 
sig 
   type 'a t 
   exception Empty 
   val create: unit -> 'a t 
   val push: 'a -> 'a t -> unit 
   val pop: 'a t -> 'a 
end

\end{verbatim}
Definiraj modul Queue, ki s pomo\v cjo dveh skladov implementira vrsto. Prvi sklad predstavlja za\v cetek vrste in drugi sklad predstavlja konec vrste. Implementiraj funkciji enqueue and dequeue !

Namig: \v ce je kateri izmed skladov prazen potem lahko drugega "obrnemo". 


\end{ex} \begin{ex}
Tip ('a*'b) list opisuje seznam parov kjer je prva komponenta klju\v c tipa 'a in druga komponenta vrednost tipa 'b. 

Napi\v si funkcijo 

meet : ('a*'b) list -> ('a*'b) list -> ('a*('b*'b)) list,

ki zdru\v zi dva seznama sortirana po klju\v cu tipa 'a. Pari iz vhodnih seznamov se zdru\v zijo samo v primeru, da se ujemajo v klju\v cu. Rezultat funkcije so pari sestavljeni iz klju\v ca in vrednosti, ki vsebuje obe vrednosti iz vhodnih parov.  

Primer:    

meet [(1,2);(2,3);(4,5);(4,9)] [(2,4);(4,6)] -> 
     [(2,(3,4));(4,(5,6));(4,(9,6))]


\end{ex} \begin{ex}
Dano imamo sortirano polje celih \v stevil. Definiraj funkcijo encode,ki vrne polje parov kjer je prva komponenta element vhodnega polja in druga komponenta \v stevilo pojavitev elementa v polju.   

Primer:

encode [|1;1;3;4;4;5|] -> [|(1,2);(3,1);(4,2);(5,1)|]


\end{ex} \begin{ex}
Boolovi izrazi so predstavljeni z naslednjim rekurzivnim tipom.   

\begin{verbatim}
type bool_exp =
   | Val of bool
   | Not of bool_exp
   | And of bool_exp * bool_exp
   | Or of bool_exp * bool_exp;

\end{verbatim}
Napi\v si funkcijo
\begin{verbatim}

eval : bool_exp -> bool, 
\end{verbatim}

ki evaluira boolov izraz v vrednost.   

Dodatna naloga: Napi\v si funkcijo, ki izpi\v se pravilnostno tabelo
za dan boolov izraz.


\end{ex} 
\begin{ex} 
  a) Definiraj razred matrix za predstavitev matrik, ki shranjujejo
  elemente poljubnega tipa 'a.

  1. Argumenti razreda naj definirajo dimenzije matrike in za\v cetno
  vrednost elementov.

  2. Napi\v si metodo set : int*int -> 'a -> unit, kjer prvi parameter
  predstavlja indekse elementa in drugi parameter hrani novo vrednost
  indeksiranega elementa.

  3. Napi\v si metodo get : int*int -> 'a, ki vrne vrednost elementa
  indeksiranega s prvim parametrom metode.
       
  b) Definiraj razred int\_matrix kot podrazred razreda matrix.

  c) V razredih matrix in int\_matrix definiraj metodo equals :
  int*int -> int*int -> bool, ki primerja dva elementa matrike in vrne
  boolovo vrednost true, \v ce sta enaka in false sicer.


\end{ex} 
\begin{ex}
  Tip ('a*'b) list opisuje seznam parov, kjer je prva komponenta para
  klju\v c tipa 'a in druga komponenta vrednost tipa 'b.

  Predpostavimo, da so vhodni seznami sortirani po klju\v cu tipa
  'a.Napi\v si funkcijo

\begin{verbatim}
diff : ('a*'b) list -> ('a*'b) list -> ('a*'b) list,
\end{verbatim}

  ki izra\v cuna razliko dveh seznamov. Rezultat vsebuje vse pare iz
  prvega seznama s klju\v ci, ki se ne pojavijo v drugem seznamu.

\noindent\/Primer:    
\begin{verbatim}
diff [(1,2);(2,3);(4,5)];(5,6)] [(2,4);(4,6)] -> [(1,2);(5,6)]
\end{verbatim}


\end{ex} 
\begin{ex}
  Tip text definira predstavitev teksta v urejevalniku
  besedil. Primerek tipa text je sestavljen iz vrstic, ki so
  sestavljene iz besed.

\begin{verbatim}
type text = Eot | Line of line * text
and line = Eol | Word of string * line
\end{verbatim}

  Napi\v si funkcijo search : text -> line -> bool, ki vrne true, \v
  ce je sekvenca besed definirana z drugim parametrom pod-sekvenca v
  tekstu, ki je podan kot prvi parameter. Predostavimo, da je iskana
  sekvenca vedno v eni vrstici.


\end{ex} 
\begin{ex}
  Splo\v sno drevo 'a tree je definirano na naslednji na\v cin.

\begin{verbatim}
type 'a tree = { mutable key:'a; mutable trees: 'a tree }
\end{verbatim}

  Napi\v si funkcijo tree\_apply : 'a tree -> ('a -> 'b) -> 'b
  tree. Klju\v ci vozli\v s\v c vhodnega drevesa se zamenjajo z
  vrednostmi funkcije (definirane z drugim parametrom) aplicirane na
  klju\v cu vozli\v s\v ca.

\end{ex} 
\begin{ex}
  Kalkulatorji, ki uporabljajo obrnjeno poljsko notacijo shranjujejo
  operande na skladu.

  Operacije kalkulatorja so plus, minus, mult in divide. Vsaka
  operacija vzame iz sklada vrhnje dve vrednosti, izra\v cuna
  operacijo in vrne rezultat na vrh sklada.

  a) Definiraj abstraktni razred rpc, ki implementira kalkulator
  osnovan na obrnjeni poljski notaciji. Vrednosti s katerimi dela
  kalkulator naj bodo poljubnega tipa 'a. Abstraktni razred rpc naj
  implementira tudi operaciji push in pop. Bodi pozor-na/-en na
  virtualne oz. konkretne metode !

  b) Definiraj konkreten razred int\_rpc kot implementacijo razreda
  rpc for za konkreten tip 'a=int.


\end{ex} 
\begin{ex}
  Seznam tipa (string*string) list predstavlja lete neke letalske
  dru\v zbe. Vsak par opi\v se direkten let med dvemi mesti. Napi\v si
  funkcijo

\begin{verbatim}
povezava : (string*string) -> (string*string) list -> bool,
\end{verbatim}

  ki preveri ali obstaja povezava med dvemi mesti podanimi s prvim
  parametrom. Povezava je lahko direktna ali z enim vmesnim
  postankom. Seznam letov letalske dru\v zbe je podan z drugim
  parametrom. Funkcija vrne true v primeru, da povezava obstaja in
  false sicer.

  Dodatne to\v cke: Naj povezava pomeni poljubno povezavo, ki je
  sestavljena iz poljubnega \v stevila vmesnih postankov.


\end{ex} 
\begin{ex}
  Naloga je sestavljena iz dveh delov:

\begin{itemize}
\item Definiraj parametri\v cni tip ('a,'b) key\_val, ki predstavlja
  zapis z dvema imenovanima komponentama:
  \begin{itemize}
  \item komponente key tipa 'a in
  \item komponente value tipa 'b.
  \end{itemize}
\item Napi\v si polimorfi\v cno funkcijo vi\v sjega reda

\begin{verbatim}
array_filter : ('a,'b) key_val array -> ('a -> bool) ->  
               ('a,'b) key_val array
\end{verbatim}

  ki iz polja podanega s prvim parametrom prepi\v se v kon\v cno polje
  samo tiste zapise s klju\v cem tipa 'a, za katere funkcija podana z
  drugim parametrom vrne true. \end{itemize}



\end{ex}
\begin{ex}
  Drevo int\_tree je definirano na naslednji na\v cin.

\begin{verbatim}
type int_tree = { mutable key:int; mutable trees: int_tree list}
\end{verbatim}

  Napi\v si funkcijo vi\v sjega reda

\begin{verbatim}
tree_filter : int_tree -> int -> int list.
\end{verbatim}

  Funkcija vrne seznam vseh klju\v cev iz drevesa podanega s prvim
  parametrim, ki so ve\v cji od vrednosti podane z drugim parametrom
  funkcije.


\end{ex} 
\begin{ex}
  a) Definiraj razred vmesnik, ki implementira FIFO vrsto elementov
  poljubnega tipa 'a. Razred naj vsebuje dve metodi:

  1) dodajanje n novih elementov na konec vrste, in
       
  2) odvzemanje k elementov iz za\v cetka vrste.

  b) Na osnovi razreda vmesnik definiraj podrazred float\_vmesnik, ki
  implementira vmesnik elementov tipa float.



\end{ex} 
\begin{ex}
  a) Definiraj tip plist, ki predstavi seznam parov. Prva komponenta
  para vsebuje klju\v ce tipa int in druga komponenta vrednosti tipa
  string.

  b) Napi\v si funkcijo vapply : plist -> (string -> string) -> plist,
  ki aplicira funkcijo tipa string -> string (drugi parameter) na vseh
  drugih komponentah parov, ki so podani s prvim parametrom
  funkcije. 

  Rezultat naj bo seznam parov kjer jer druga komponenta vhodnih parov
  zamenjana z vrednostjo funkcije string -> string.


\end{ex} 
\begin{ex}
  Tekst urejevalnika je shranjen kot polje seznamov besed:

\begin{verbatim}
type text = string list array;;
\end{verbatim}
  
  Definiraj funkcijo find\_replace : text -> string -> string -> text,
  ki zamenja vse pojavitve niza podanega z drugim parametrom z nizom
  znakov podanim s tretjim parametrom v tekstu podanim s prvim
  parametrom. Rezutat je tekst z zamenjanim nizom.


\end{ex} 
\begin{ex}
  Drevo a\_tree je definirano na naslednji na\v cin.

\begin{verbatim}
type 'a a_tree = { mutable key:'a; mutable trees: a_tree list}
\end{verbatim}

  Napi\v si funkcijo vi\v sjega reda

\begin{verbatim}
tree_filter : 'a a_tree -> ('a -> bool) -> 'a list.
\end{verbatim}

  Funkcija vrne seznam vseh klju\v cev iz drevesa podanega s prvim
  parametrim, za katere vrne funkcija podana z drugim parametrom
  vrednost true.


\end{ex} 
\begin{ex}
  Definiraj razred queue, ki implementira obi\v cajno vrsto z uporabo
  polja. Vrsta naj vsebuje elemente tipa int. Razred naj vsebuje
  naslednje metode.

  \textbf{enqueue:} Elemente dodajajamo po vrsti v polje. V primeru,
  da pridemo do konca polja za\v cnemo dodajati spet na za\v
  cetku. Pri dodajanju preveri ali je vrsta \v ze polna.
                 
  \textbf{dequeue:} Elemente jemljemo po vrsti iz za\v cetka polja. Ko
  pridemo do konca polja, za\v cnemo od za\v cetka. Pri jemanju
  elementov iz vrste preveri ali je vrsta prazna.

  Velikost vrste naj bo parameter razreda, ki se takoj pri kreaciji
  objekta zaokr\v zi navzgor na KB.

  Inicializator razreda naj takoj po kreaciji objekta vstavi v vrsto
  10 ni\v cel.


\end{ex}
\begin{ex}
Napi\v si parametri\v cno funkcijo 

\begin{verbatim}
podseznam : 'a list -> int*int -> 'a list, 
\end{verbatim}

ki iz danega seznama izlu\v s\v ci podseznam dolo\v cen z 2. parametrom funkcije (z,k), 
kjer z predstavlja indeks za\v cetka podseznama in k indeks konca podseznama. 

Predpostavimo, da je $z<k$ in $k<l$, kjer je $l$ dol\v zina vhodnega seznama. 

\end{ex} 

\begin{ex}
Definiraj modul, ki zna delati z enodimenzionalnimi polji poljubnega (!)
tipa. Tip elementa polja je torej spremenljivka tipa.e

\textbf{Funkcije}: kreiranje, spajanje, uni\v cevanje, odvzem.
\end{ex} 

\begin{ex}
Naredi modul "garaza", ki:

a) odpira in zapira vrata gara\v ze z enim samim ukazom (\v ce so vrata
odprta jih zapre, sicer naj jih odpre);

b) parkira (odpelje/pripelje) avtomobile ali motorje, pri \v cemer je
\v stevilo vozil omejeno, motor pa zasede polovico prostora avtomobila;

c) izpi\v se \v stevilo in tip vozil v parkiranih v gara\v zi.
\end{ex}

