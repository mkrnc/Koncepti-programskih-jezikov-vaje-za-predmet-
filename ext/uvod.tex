\chapter{Uvod}

\noindent
Zakaj u\v cenje konceptov programskih jezikov? \\
Posplo\v sevanje abstrakcij podanih v konkretnih jezikih \\
Novi predmeti na podro\v cju programskih jezikov \\
%``Multipardigm'' programski jeziki Scala, C#, F#, ... \\

\noindent
Osnovna ideja strukture zbirke? \\
Prva klasifikacija podro\v cij so modeli programskih jezikov \\
Znotraj modelov so pomembni koncepti \\
\v student naj bi preko nalog ponovil/obnovil poznavanje posameznih konceptov \\

\noindent
Osnovne ideje predstavitve podro\v cja \\
U\v citi se ve\v cih PJ? \\
\v sola programskih jezikov bli\v zje teoriji programskih jezikov \\
Pri predstavitvi smo bli\v zje matemati\v cni, logi\v cni razlagi \\
U\v cenje o abstrakcijah in o implentaciji PJ \\

\noindent
Zakaj OCaml? \\
Meta Language ML, trenutni trendi v na\v crtovanju jezikov \\ 
Osnova \v cisti lambda ra\v cun \\
OCaml pokriva \v stiri razli\v cne modele \\

\noindent
Kak\v sne vrste nalog imamo?
Temeljni/klasi\v cni problemi iz danegih modelov programskih jezikov \\
Primer, imperativni jeziki, delo s polji \\
Matemati\v cni problemi, blizu teorije programskih jezikov \\

\noindent
Re\v sitve so podane v OCaml \\
Lahko uporabimo tudi primeren jezik iz danega modela jezikov? \\



