\chapter{Uvod}

\noindent
Zakaj učenje konceptov programskih jezikov? \\
Posploševanje abstrakcij podanih v konkretnih jezikih \\
Novi predmeti na področju programskih jezikov \\
%``Multipardigm'' programski jeziki Scala, C#, F#, ... \\

\noindent
Osnovna ideja strukture zbirke? \\
Prva klasifikacija področij so modeli programskih jezikov \\
Znotraj modelov so pomembni koncepti \\
Študent naj bi preko nalog ponovil/obnovil poznavanje posameznih konceptov \\

\noindent
Osnovne ideje predstavitve področja \\
Učiti se večih PJ? \\
Šola programskih jezikov bližje teoriji programskih jezikov \\
Pri predstavitvi smo bližje matematični, logični razlagi \\
Učenje o abstrakcijah in o implentaciji PJ \\

\noindent
Zakaj OCaml? \\
Meta Language ML, trenutni trendi v načrtovanju jezikov \\ 
Osnova čisti lambda račun \\
OCaml pokriva štiri različne modele \\

\noindent
Kakšne vrste nalog imamo?
Temeljni/klasični problemi iz danegih modelov programskih jezikov \\
Primer, imperativni jeziki, delo s polji \\
Matematični problemi, blizu teorije programskih jezikov \\

\noindent
Rešitve so podane v OCaml \\
Lahko uporabimo tudi primeren jezik iz danega modela jezikov? \\



