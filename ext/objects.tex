\chapter{Objektno orientirani programski jeziki}



\section{Definicija razredov}



\begin{ex}
Definiraj naslednja dva razreda za delo s celimi \v stevili.

\begin{enumerate}
\item
Cela \v stevila bi radi obravnavali kot objekte. Definiraj osnovno aritmetiko za delo s celimi \v stevili: se\v stevanje, od\v stevanje, celo\v stevilsko deljenje in mno\v zenje.
\item
Pozitivna cela \v stevila skupaj z ni\v clo so poseben primer celih \v stevil, ki jih imenujemo naravna \v stevila. Vse operacije prilagodi tako, da v primeru, da je parameter operacije negativen, operacija najprej zamenja parameter z absolutno vrednostjo, ga pomno\v zi s 100 ter \v sele nato opravi \v zeljeno operacijo.
\end{enumerate}
Napi\v si tudi konstruktorja celih in naravnih \v stevil. Bodi pozoren na uporabo dedovanja pri konstrukciji re\v sitve.
\end{ex}





\begin{ex}
Definiraj razred, ki vsebuje polje celih \v stevil sortirano po velikosti. Napi\v si metodo \lstinline{zdruzi : int array -> int array}, ki pridru\v zi polju celih \v stevil na\v sega razreda \v ze sortirano polje a tako, da je rezultat sortiran.
\end{ex}




\begin{ex}
Definirajte razred matrika s katerim predstavimo dvodimenzionalno matriko realnih \v stevil. Katere atribute potrebujemo?
\begin{enumerate}
\item Definiraj bralce in pisalce razreda matrika ter kodo s katero se inicializira na novo kreirana matrika.
\item Napi\v site metodo \lstinline{invertirajx2 : unit}, ki invertira matriko po obeh diagonalah.
\end{enumerate}
\end{ex}





\begin{ex}
Firma iz avtomobilske industrije bi \v zelela predstaviti motorje v programskem jeziku Ocaml. Podatkovna struktura naj predstavi hierarhi\v cno kompozicijo motorja iz komponent. 

Vsaka komponenta motorja ima identifikator, ime, stanje in seznam pod-komponent, ki jih predstavimo spet kot komponente. Imamo \v se naslednje podatke.
\begin{itemize}
\item Stanje komponente pove ali je komponenta delujo\v ca z boolovo vrednostjo (Dela|Nedela). 
\item Celoten motor je predstavljen kot komponenta. 
\item Osnovne komponente (osnovni deli) nimajo pod-komponent.
\end{itemize}
\begin{enumerate}
\item Definiraj objektno predstavitev motorja oz. komponent motorja.
\item Predpostavimo, da imamo na voljo funkcijo \lstinline{test : boolean}, ki za dano osnovno! komponento (identifikator) vrne vrednost boolovo vrednost (Dela|Nedela). 
\end{enumerate}
Napi\v si funkcijo \lstinline{oznaci : unit}, ki ozna\v ci vse komponente motorja z Dela oz. Nedela. Edino pravilo, ki ga moramo upo\v stevati: komponenta, ki ima vse delujo\v ce pod-komponente je tudi sama delujo\v ca (Dela) sicer pa ni delujo\v ca (Nedela). 
\end{ex}








\section{Hierarhije razredov}





\begin{ex}
Vrsto (FIFO) in sklad (LIFO) celih \v stevil \v zelimo implementirati z uporabo polja. 

Najprej definiraj razred \lstinline{zbirka}, ki realizira vrsto kjer lahko dodajamo in odvzemamo elemente na za\v cetku in na koncu vrste. 
Razreda \lstinline{vrsta} in \lstinline{sklad} implementiraj kot specializaciji razreda zbirka. 

\begin{itemize}
\item Vrsta naj ima operaciji \texttt{enqueue} in \texttt{dequeue}. Prva operacija doda nov element na za\v cetek vrste in druga odvzame element iz konca vrste. 

\item Sklad naj ima operaciji \texttt{push} in \texttt{pop}. Operacija \texttt{push} doda nov element na vrh sklada in \texttt{pop} odvzame elemet iz vrha sklada.
\end{itemize}
\end{ex}





\begin{ex}
Definiraj hierarhijo razredov za predstavitev geometrijskih objektov: to\v cka, premica in krog. 

Vsak geometrijski objekt naj ima funkcijo
\begin{lstlisting}
premakni_se : int -> int -> unit,
\end{lstlisting}
kjer predstavljata parametra premika po $x$ in $y$ osi. 
\end{ex}






\begin{ex}
Roboti \v zivijo v fiksnem vnaprej definiranem dvo-dimenzionalnem svetu to\v ck. Svet ima koordinati x=1..100 in y=1..100. Meje sveta so vgrajene v robote. Na svetu imamo tri vrste robotov.

\begin{enumerate}
\item Osnovni robot, ki je definiran z za\v cetno to\v cko x,y in odmikom dx po osi x in odmikom dy po osi y, ki ga naredi ob premiku. Ko pride do konca sveta po koordinati x se dx zamenja z -dx in ko prispe do konca sveta po koordinati y se dy zamenja z -dy.

\item Ponikajo\v ci robot, ki se premika na enak na\v cin kot osnovni robot le da vsakih k premikov ponikne in se ne nari\v se.

\item Cikcak robot, ki se prav tako premika enako kot osnovni robot le da \v se ska\v ce po x osi dve to\v cki od premika osnovnega robota na levo in v naslednjem koraku dve to\v cki na desno in tako naprej.
\end{enumerate}

Definiraj svet robotov v programskem jeziku Ocaml. 
\begin{itemize}
\item Vsak robot naj se predstavi ob kreaciji, tako da uporabnik vidi katerim razredom robot pripada.

\item Napi\v si metodo \lstinline{premik : unit}, ki premakne robota na naslednjo pozicijo v odvisnosti od vrste robota. Metoda \texttt{premik} najprej izra\v cuna naslednjo pozicijo robota in jo nato izpi\v se (t.j. izpi\v se to\v cko x,y) oz. je tudi ne izpi\v se v primeru ponikajo\v cega robota.
\end{itemize}
\end{ex}





\begin{ex}
V dvo-dimenzionalnem svetu robotov imamo dve vrsti robotov: x-robota, ki se premika samo po x-osi in y-robota, ki se premika samo po y-osi. Svet je definiran na področjih [-10..10] po x-osi in [-10..10¸] po y-osi. x-robot je definiran na lokaciji, ki je na y-osi, medtem ko ima y-robot lokacijo na x-osi. 

Premik robota po x-osi implementiramo tako, da pri\v stejemo x-koordinati 1 oz. -1, odvisno od tega ali se robot premika desno ali levo. Ko robot pride do roba sveta, se obrne v nasprotno smer. Premikanje robota po y-osi je definirano enako kot v primeru premikanja po x-osi le da so osi zamenjane in se robot premika navzgor in navzdol.

\begin{enumerate}
\item Definiraj razrede s katerimi predstavimo x-robota in y-robota. Uporabi skupen koren hierarhije razredov \lstinline{robot}, ki realizira skupne značilnosti robotov.

\item Vsi razredi naj imajo definirano metodo \lstinline{premik : unit}. 
\end{enumerate}
\end{ex} 





\begin{ex}
Definirati je potrebno hierarhijo razredov za predstavitev podatkov o \v studentih v informacijskem sistemu \v SIS. Splo\v sne podatke o \v studentih predstavimo v razredu Oseba. Bolj specifi\v cne podatke predstavimo v razredih Student, PodiplomskiStudent in Asistent.
Asistent je poseben primer podiplomskega \v studenta. 

V razredih bi \v zeleli hraniti naslednje podatke: ime in priimek, naslov, tel.\v stevilka, vpisan letnik, vpisan program, obstoje\v ca izobrazba in povpre\v cna ocena. 

Naloga ima tri dele:

\begin{enumerate}
\item Definiraj razrede v programskem jeziku Ocaml.

\item Realiziraj razrede tako, da inicializatorji razreda inicializirajo objekt z za\v cetnimi vrednostmi.

\item V hierarhiji razredov implementiraj metodo \lstinline{predstavi :unit}, ki predstavi vse lastnosti poljubnega primerka razredov v hierarhiji. Uporabljaj dedovanje in prekrivanje metod!
\end{enumerate}
\end{ex}




%\begin{ex}
%Hi\v sa je sestavljena iz N nadstropij in podstre\v sja. Vsako nadstropje ima M sob.
%
%V vsaki sobi in na podstre\v sju imamo termometer. Temperaturo hi\v se dolo\v cimo tako, da izra\v cunamo povpre\v cje meritev temperature v vseh sobah in na podstre\v sju. 
%
%Predpostavimo, da imamo dano funkcijo 
%\begin{lstlisting}
%temperatura : int -> int -> int,
%\end{lstlisting} 
%ki za dano nadstropje in \v stevilko sobe vrne vrednost temperature v stopinjah. Podstre\v sje identificiramo kot N+1 nadstropje (soba=0).
%
%\begin{enumerate}
%\item Definiraj razrede s katerimi predstavimo: sobe, nadstropja, podstre\v sje in celotno hi\v so.
%\item Vsak razred mora imeti metodo \lstinline{odcitajTemperaturo : int}, ki vrne temperaturo objekta in postavi trenutno vrednost temperature za dan objekt.
%\item Razred \lstinline{Hisa} mora izra\v cunati temperaturo hi\v se po zgoraj opisanem postopku.
%\end{enumerate}
%\end{ex}



\begin{ex}
Hi\v sa je sestavljena iz \texttt{N} nadstropij. Vsako nadstropje ima \texttt{M} sob. V vsaki sobi imamo termometer. Temperaturo hi\v se dolo\v\/cimo tako, da izra\v cunamo povpre\v cje meritev temperature v vseh sobah. 

Predpostavimo, da imamo dano funkcijo \lstinline{temperatura : int -> int -> int}, ki za dano nadstropje in \v stevilko sobe vrne vrednost temperature v stopinjah. 

\begin{itemize}
\item Predstavi opisano okolje z razredi \lstinline{soba}, \lstinline{nadstropje} in \lstinline{hisa}.

\item Vsak razred mora imeti metodo \lstinline{odcitaj_temp : int -> int -> int}, ki vrne temperaturo objekta in  postavi trenutno vrednost temperature za dan objekt.

\item Razred \lstinline{hisa} naj vsebuje metodo \lstinline{povprecna_temp : int}, ki izra\v cuna povpre\v cno teperaturo hi\v se po zgoraj opisanem postopku.
\end{itemize}
\end{ex} 








\section{Rekurzivne strukture objektov}



\begin{ex}
Dano je drevo, ki je definirano z razredom \lstinline{vozel}. Vsako vozli\v s\v ce hrani vrednost klju\v ca \lstinline{kljuc}. Za predstavitev pod-dreves uporabimo tip \lstinline{'a option}.
\begin{lstlisting}
class vozel k = 
object
  val kljuc = k
  val levo = None
  val desno = None
end
\end{lstlisting}

Napi\v si metodo \lstinline{preveriUrejenost : boolen}, ki preveri ali je dano drevo urejeno. V danem korenu imajo vsa vozli\v s\v ca v levem poddrevesu manj\v so vrednost klju\v ca in vsa vozli\v s\v ca v desnem poddrevesu ve\v cjo vrednost klju\v ca.

Dodatna naloga: Kako bi sproti pri preverjanju urejenosti ponovno uredil drevo, \v ce ima\v s dano metodo za vstavljanje v urejeno drevo?
\end{ex}






\begin{ex}
Dano je drevo, ki je definirano z razredom \lstinline{vozel}. Vsako vozli\v s\v ce hrani vrednost klju\v ca \lstinline{kljuc}. Za predstavitev pod-dreves uporabimo tip \lstinline{'a option}.
\begin{lstlisting}
class vozel k = 
object
  val kljuc = k
  val levo = None
  val desno = None
end
\end{lstlisting}

\begin{enumerate}
\item Napi\v si metodo \lstinline{vsota : 'a}, ki za dano vozli\v s\v ce izra\v cuna vsoto vseh klju\v cev vozli\v s\v c pod-drevesa.

\item Izpi\v si korene poddreves s vsoto < 10.
\end{enumerate}
\end{ex}






\begin{ex}
Naprave so sestavljene iz manj\v sih naprav, ki so spet lahko sestavljene iz manj\v sih naprav, itd. Vsaka naprava ima svoje ime, tip in te\v\/zo.
\begin{enumerate}[label=(\Alph*)]
\item Definiraj razred \lstinline{naprava}. Bodi pozor-na/en na definicijo parametrov razreda in na inicializacijo razreda. 

\item Definiraj metodo \lstinline{prestej : int}, ki za dano napravo vrne število vseh komponent. 

\item Definiraj metodo \lstinline{listi : naprava list} razreda \lstinline{naprava}, ki v seznamu vrne liste drevesa vsebovanosti naprav za dano napravo (objekt).
\end{enumerate}
\end{ex} 








\section{Abstraktni razredi}


\begin{ex}
Geometrijski objekt je definiran z virtualnim razredom \lstinline{geo}, ki vsebuje definiciji naslednjih virtualnih metod.

\begin{lstlisting}
method virtual predstavi : string
method virtual narisi : unit 
\end{lstlisting}

Metoda \lstinline{predstavi} vrne niz znakov s katerim je opisan geometrijski objekt. Metoda \lstinline{narisi} nari\v se dani geometrijski objekt. 

Definiraj abstraktni razred \lstinline{geo} in razrede \lstinline{tocka}, \lstinline{krog} in \lstinline{premica} kot
implementacije abstraktnega razreda \texttt{geo}. Uporabi dedovanje kjer je mogo\v ce. 

Implementiraj metodi predstavi in narisi za vse tri konkretne razrede. Predpostavi, da imamo \v ze napisane naslednje funkcije.
\begin{lstlisting}
narisi_tocko : int*int -> unit
narisi_premico : int*int -> int*int -> unit
narisi_krog : int*int -> int -> unit
\end{lstlisting}

Uporabi prekrivanje metod in kodo razporedi tako, da bolj specifi\v cne metode uporabijo prekrite metode, kjer je mogo\v ce.
\end{ex}





\begin{ex}
Definiraj hierarhijo razredov za geometrijske objekte to\v cko, krog in kvadrat. Poskusi v \v cim ve\v cji meri uporabiti gradnike objektnega modela za u\v cinkovito predstavitev hierarhije. 

\begin{enumerate}
\item Definiraj razrede \lstinline{tocka}, \lstinline{krog} in \lstinline{kvadrat}. Uporabi abstraktni razred za vrh hierarhije razredov.

\item Za vse geometrijske objekte definiraj metodo \lstinline{premakni}, ki premakne dan geometrijski objekt. Uporabi dedovanje in prekrivanje metod!

\item Opi\v si potek inicializacije objekta ob kreiranju novega objekta.
\end{enumerate}
\end{ex} 







\section{Parametri\v cni razredi}





\begin{ex}
Implementiraj razred class \lstinline{'a seznam}, ki naredi ovojnico okoli vgrajenega tipa \texttt{list}. Implementiraj metode, ki realizirajo obi\v cajne operacije nad seznami vključno z operacijo \texttt{::} (\texttt{cons}), operacijo \texttt{@} in testom vsebovanosti \texttt{member : 'a -> 'a seznam -> boolean}.
\end{ex}





\begin{ex}
Definiraj parametriziran razred \lstinline{'a polje}, ki realizira ovojnico okoli tipa \lstinline{array}. Razred \texttt{'a polje} naj ima definirane naseldnje metode.
\begin{lstlisting}
method popravi : int -> 'a -> unit
method preberi : 'a 
\end{lstlisting}
Metoda \texttt{popravi} za dan indeks in vrednost tipa 'a postavi vsebino elementa polja z danim indeksom na novo vrednost. Metoda \texttt{preberi} vrne za dan indeks vsebino elementa polja. 

Polje naj bo kreirano ob kreaciji objekta. Ob kreaciji podamo tudi velikost polja.

Z uporabo prej definiranega razreda \lstinline{polje} definiraj razred \lstinline{real_polje}, ki vsebuje realna \v stevila. Re-definiraj metodo preberi tako, da vrne samo celo vrednost realnega \v stevila.
\end{ex}





\begin{ex}
Definiraj razred \lstinline{queue}, ki implementira obi\v cajno vrsto z uporabo polja. Vrsta naj vsebuje elemente tipa \texttt{'a}. Razred naj vsebuje naslednje metode.

\begin{lstlisting}
enqueue : 'a -> unit
dequeue : 'a
\end{lstlisting}

Metoda \texttt{enqueue} dodaja elemente po vrsti v polje. V primeru, da pridemo do konca polja za\v cnemo dodajati spet na za\v cetku. Pri dodajanju preveri ali je vrsta \v ze polna.
           
Metoda \texttt{dequeue} jemlje elemente po vrsti iz za\v cetka polja. Ko pridemo do konca polja, za\v cnemo od za\v cetka. Pri jemanju elementov iz vrste preveri ali je vrsta prazna.

Velikost vrste naj bo parameter razreda, ki se takoj pri kreaciji objekta zaokro\v zi navzgor na KB.

Inicializator razreda naj takoj po kreaciji objekta vstavi v vrsto 10 ni\v cel.
\end{ex}








\begin{ex}
\begin{enumerate}
\item Definiraj razred \lstinline{zbirka}, ki hrani zbirko vrednosti tipa \texttt{'a}. Implementiraj naslednje metode.
\begin{lstlisting}
dodaj_zacetek : 'a -> unit
beri_zacetek : 'a 
dodaj_konec : 'a -> unit
beri_konec : 'a  
\end{lstlisting}

\item Razreda \lstinline{vrsta} (FIFO) in \lstinline{sklad} (FILO) implementiraj kot specializaciji razreda \lstinline{zbirka}. Definiraj ustrezne metode razredov \lstinline{vrsta} (\texttt{enqueue} in \texttt{dequeue}) in \lstinline{sklad} (\texttt{push} in \texttt{pop}).
\end{enumerate}
\end{ex} 







\begin{ex}
Slika naj bo predstavljena s trojicami, ki vsebujejo x in y koordinati ter barvo.

\begin{enumerate}
\item Definiraj parametriziran abstraktni razred \lstinline{'a tocka}, s katero definiramo to\v cke predstavljene z dvema koordinatama tipa int in barvo tipa 'a. 

Definiraj abstraktno metodo \lstinline{enakost : 'a tocka -> bool}, ki primerja dano to\v cko s to\v cko, ki je podana kot parameter metode \texttt{enakost}.

Definiraj konkreten razred \lstinline{itocka}, kjer je tip barve celo \v stevilo (\texttt{int}). Razred \texttt{itocka} podeduje vse lastnosti razreda \texttt{tocka} in implementira metodo \texttt{enakost}.

\item Z uporabo prej definiranega abstraktnega razreda \texttt{'a tocka} definiraj parametriziran razred \lstinline{'a slika}, kjer tip \texttt{'a} spet predstavlja tip barve to\v ck slike. 

Skiciraj konkretizacijo parametriziranega razreda \lstinline{'a slika} v razred \lstinline{intslika}, kjer je spremenljivka tipa \texttt{'a} enaka tipu \texttt{int}.
\end{enumerate}
\end{ex}







%--- nekateg.
\begin{ex}
V svetu eno-dimenzionalnih robotov se roboti pomikajo po premici levo in desno. Premica ima izhodi\v s\v ce, ki ima vrednost 0. Roboti se lahko pomikajo za korak levo ali korak desno. 

Imamo dve vrsti robotov, ki se premikajo po podro\v cju [-100..100]:

\begin{enumerate}
\item Robot se premakne za eno enoto v izbrano smer. Ko pride do meje se obrne. To vrsto robota predstavimo z razredom \lstinline{robot1}.

\item Robot se premika naklju\v cno za eno enoto desno ali levo. To vrsto robota predstavimo z razredom \lstinline{robot2}.
\end{enumerate}

\begin{enumerate}[label=(\roman*)]
\item Definiraj robote z razredi povezanimi v hierarhijo dedovanja. Vrh hierarhije dedovanja naj bo razred \lstinline{robot}.

\item Definiraj za vse robote metodo \lstinline{premik}, ki naredi naslednji premik robota.

\item Definiraj razred \lstinline{robot3} kot podrazred razreda \lstinline{robot1}. Z uporabo prekrivanja definiraj premik tega robota kot dvakraten premik robota iz razreda \lstinline{robot1}.
\end{enumerate}
\end{ex} 




%\begin{ex}
%Imamo naprave, ki so sestavljene iz manj\v sih naprav, ki so spet lahko sestavljene iz manj\v sih naprav, itd. Vsaka naprava ima svoje ime in %tip. Naprave definiramo z naslednjim razredom \lstinline{naprava}.
%
%\begin{lstlisting}
%class naprava (im:string) (tp:string) (ln:naprava list) = 
%object 
%  val ime = im 
%  val tip = tp 
%  val mutable komponente = ln 
%end 
%\end{lstlisting}
%
%Napi\v si metodo \lstinline{izpisi_liste}, ki izpi\v se imena in tipe vseh listov--naprav, ki nimajo nobene komponente.
%\end{ex} 



\begin{ex}
\begin{enumerate}[label=(\Alph*)]
\item Definiraj razred \lstinline{matrika}, ki predstavlja \texttt{mxn} matrike realnih \v stevil. Razred naj se inicializira s parametroma \texttt{m} in \texttt{n}.

\item Definiraj naslednje metode razreda \lstinline{matrika}.
\begin{lstlisting}
get : int -> int -> float
set : int -> int -> float -> unit
mul : matrika -> unit
\end{lstlisting}

Vse metode spreminjajo matriko predstavljeno z objektom, ki izvr\v si metodo.  Na primer, metoda mul pomno\v zi dano matriko (objekt) z matriko, ki je podana kot parameter. Rezultat je shranjen v matriko (objekt), ki izvr\v si metodo.

\item Definiraj podrazred \lstinline{kmatrika}, ki predstavlja kvadratno matriko \texttt{nxn}.
\end{enumerate}
\end{ex} 



\begin{ex}
Kalkulatorji, ki uporabljajo obrnjeno poljsko notacijo shranjujejo operande na skladu.

Definiraj abstraktni razred \lstinline{rp_calc}, ki implementira kalkulator osnovan na obrnjeni poljski notaciji. Uporabnik komunicira z kalkulatorjem preko vmesnika \lstinline{rp_calc}. Vrednosti s katerimi dela kalkulator naj bodo poljubnega tipa 'a. 

Operacije kalkulatorja so \texttt{plus}, \texttt{minus}, \texttt{mult} in \texttt{divide}. Vsaka operacija vzame iz sklada vrhnje dve vrednosti, izra\v cuna operacijo in vrne rezultat na vrh sklada.

Razred \lstinline{rp_calc} naj implementira tudi operaciji \texttt{push} in \texttt{pop}. 

Definiraj konkreten razred \lstinline{int_rpc} kot implementacijo razreda \lstinline{rp_calc} for za konkreten tip \texttt{'a=int}.
\end{ex} 





%\begin{ex}
%Naloga je sestavljena iz dveh delov:
%
%  a) Definiraj razred vmesnik, ki implementira FIFO vrsto elementov
%  poljubnega tipa 'a. Razred naj vsebuje dve metodi:
%
%  1) dodajanje n novih elementov na konec vrste, in
%       
%  2) odvzemanje k elementov iz za\v cetka vrste.
%
%  b) Na osnovi razreda vmesnik definiraj podrazred float\_vmesnik, ki
%  implementira vmesnik elementov tipa float.
%\end{ex} 






\begin{ex}
Dan je razred \lstinline{sklad}, ki realizira obi\v cajen sklad celih \v stevil.
\begin{lstlisting}
class Sklad =
object 
     method size : int
     method push : int -> unit
     method pop : int
end
\end{lstlisting}

Definiraj razred \lstinline{vrsta}, ki implementira vrsto z dvema skladoma. Prvi sklad predstavlja za\v cetek vrste in drugi sklad predstavlja konec vrste. Razred Vrsta naj vsebuje operaciji: 
\begin{lstlisting}
enqueue : int -> unit, 
dequeue : int. 
\end{lstlisting}
V primeru, da je drugi sklad prazen in uporabimo operacijo \texttt{dequeue}, potem najprej obrnemo prvi sklad in ga shranimo kot drugi sklad.
\end{ex}






%*****************HERE********************
\begin{ex}
Dan je razred \texttt{Sklad}, ki realizira parametriziran sklad celih \v stevil
\begin{lstlisting}
class ['a] Sklad =
object 
     method size : 'a
     method push : 'a -> unit
     method pop : 'a
end
\end{lstlisting}
Definiraj razred Kalkulator, ki bo osnova za implementacijo enostavnega kalkulatorja, ki deluje z uporabo obrnjene poljske notacije. Kalkulator deluje na naslednji na\v cin.  

\v stevilo, ki ga vnesemo da na vrh delovnega sklada. Operacije plus, minus, mnozi in deli vzamejo zadnja dva operanda iz sklada, opravijo operacijo in vrnejo rezultat na sklad. 

Kako bi naredili splo\v sen kalkulator, ki lahko uporablja poljubno predstavitev \v stevil?


\end{ex} 



%--- parametriz. razredi
\begin{ex}
  Definiraj binarno drevo z objekti, ki predstavljajo vozli\v s\v
  ca drevesa. Definiraj parametriziran razred \lstinline{vozlisce}
  katerega parameter tip \lstinline{'a} naj predstavlja tip vrednosti
  vozli\v s\v ca.

  Pod-drevesi danega vozli\v s\v ca definiraj kot opcijske vrednosti s
  \v cimer se izognemo definiciji praznega pod-drevesa. Opcijske
  vrednosti definiramo z uporabo tipa \lstinline{'a option}.

\begin{lstlisting}
  type `a option = Some of `a | None
\end{lstlisting}

  Napi\v si metodo, ki vstavi novo vozli\v s\v ce v skrajno levo vejo
  drevesa.
\end{ex} 



\begin{ex}
  V programskem jeziku Ocaml imamo definirano podatkovno strukturo
  seznam na naslednji na\v cin:

  type 'a element = { 
     mutable vrednost:'a; 
     mutable naslednji:'a seznam 
  } 
  and 'a seznam = Prazen | Element of 'a element ;; 

  Napi\v si funkcijo zdruzi : 'a seznam -> 'b seznam -> ('a * 'b)
  seznam, ki zdru\v zi elemente dveh seznamov v en sam seznam tako, da
  prepi\v se istole\v zne elemente v pare. \v ce ne obstaja istole\v
  zni par potem naj ima komponenta vrednost () : unit.
\end{ex} 



\begin{ex}
  Napi\v si polimorfi\v cno funkcijo rapp :
  ('a->'a)->int->('a->'a). Klic rapp f n applicira funkcijo f n-krat
  na danem parametru:

  rapp f n = (function x -> f (...(f x)...))     

\noindent\/Primer:            
\begin{lstlisting}
# let f x = x+3;; 
val f : int -> int = <fun> 
# (rapp f 3) 1;; 
- : int = 10 

\end{lstlisting}

Namig: Uporabi\v s lahko funkcijo compose:

\begin{lstlisting}
# let compose f g x = f (g x) ;; 
val compose : ('a -> 'b) -> ('c -> 'a) -> 'c -> 'b = <fun> 
\end{lstlisting}
\end{ex} 



\begin{ex}
  Definiraj uporabni\v ski tip 'a btree, ki predstavlja binarno drevo
  katerega vozli\v s\v ca vsebujejo vrednost tipa 'a.

  Napi\v si funkcijo izomorfni : 'a btree -> 'a btree -> bool, ki
  preveri \v ce imata podani drevesi enako strukturo ne glede na
  vrednosti v vozli\v s\v cih.


\end{ex} 




