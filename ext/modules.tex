\chapter{Modularni programski jeziki}

%\section{Definicija modula}




\begin{ex}
Definiraj modul za delo s pari celih \v stevil. Tip par je definiran z
naslednjim stavkom: type par = int*int. Modul naj vsebuje funkcije:

\begin{lstlisting}
sestej a b: par -> par -> par
odstej a b: par -> par -> par
mnozi a b: par -> par -> par
\end{lstlisting}
Pomen operacij \emph{sestej, odstej} in \emph{mnozi} je obi\v cajen:
\begin{lstlisting}
sestej (2,3) (1,2) -> (3,5)
odstej (2,3) (1,2) -> (1,1)
odstej (2,3) (1,2) -> (2,6)
\end{lstlisting}
\end{ex}



\begin{ex}
Imlementiraj modul za delo z vrsto. V vrsto dodajamo na za\v cetek in
odvzemamo objekte na koncu vrste. Bodi pozoren/a na naslednje vidike:

- Tip elementa vrste naj se definira ob kreaciji vrste. Tip elementa
  je torej poljuben tip \lstinline{'a} in vrsta je parametri\v cni tip \lstinline{'a vrsta}.
  
- Dodaj kodo, ki bo prepre\v cevala pisanje v polno vrsto ter branje iz
  prazne vrste.
  
- Za implementacijo vrste lahko uporabi\v s poljubno podatkovno strukturo.
\end{ex}



\begin{ex}
Turisti\v cna Agencija po\v citni\v skih aran\v zmajev bo implementirala
informacijski sistem.
Definirajte modul \lstinline{Aranzma}, ki vsebuje kodo za delo z
aran\v zmaji. Aran\v zma je opisan z:

- destinacija (opisno), 

- tip\_namestitve (opisno), 

- trajanje (\v stevilo), in 

- cena (\v stevilo).

Modul mora poleg same kreacije in ogleda aran\v zmaja omogo\v cati \v se
popravljanje imena destinacije, popravljanje tipa namestitve, trajanja
in cene.

Za aran\v zma definiraj modula \lstinline{Agent} in \lstinline{Stranka}, kjer \lstinline{Agent} v aran\v zmaju
lahko pogleda aran\v zma preimenuje destinacijo, dolo\v ci nov tip
namestitve ter popravi trajanje in ceno. \lstinline{Stranka} lahko zgolj pogleda
dolo\v ceni aran\v zma.

\begin{sol}

\begin{lstlisting}
module TAgencija = 
struct
    type aranzma ={ mutable destinacija:string;
                    mutable tip_namestitve:string;
                    mutable trajanje:int;
                    mutable cena:int}
    let kreiraj dest tip trajanje cena = {destinacija=dest;
                    tip_namestitve=tip;
                    trajanje=trajanje;
                    cena=cena}
    let ogled a = a.destinacija,a.tip_namestitve, a.trajanje,a.cena
    let popravi_dest a nova_dest = a.destinacija <- nova_dest
    let popravi_t_n a nov_t_n = a.tip_namestitve <- nov_t_n
    let popravi_trajanje a novo_trajanje = a.trajanje <- novo_trajanje
    let popravi_ceno a nova_cena = a.cena <- nova_cena
end;;

module type AGENT =
  sig
    type aranzma
    val kreiraj : string -> string -> int -> int -> aranzma
    val ogled : aranzma -> string * string * int * int
    val popravi_dest : aranzma -> string -> unit
    val popravi_t_n : aranzma -> string -> unit
    val popravi_trajanje : aranzma -> int -> unit
    val popravi_ceno : aranzma -> int -> unit
  end;;
  
module type STRANKA =
  sig
    type aranzma
    val ogled : aranzma -> string * string * int * int
  end;;

module Agent = (TAgencija:AGENT with type aranzma = TAgencija.aranzma);;
module Stranka = (TAgencija:STRANKA with type aranzma = TAgencija.aranzma);;

\end{lstlisting}

\end{sol}

\end{ex}



\begin{ex}
Novo leto je za nami, vendar se \v zelimo letos \v ze zgodaj pripraviti na
naslednje novoletno praznovanje. V ta namen boste naredili modul, ki
nam bo pomagal pri izbiri jelke. Jelka naj bo sestavljena iz krosnje
in debla. Modul naj zna narediti jelko s poljubno visokim deblom in
kro\v snjo.

a) Naredi funkcijo \lstinline{ustvari()}, ki ustvari prazno jelko, funkcijo
\lstinline{povejVisino}, ki vrne vsoto vi\v sine debla in kro\v snje, ter funkciji za
nastavljanje vi\v sine debla in kro\v snje.

b) Naredi funkcijo za izris jelke (glej primer). Krosnja naj bo
sestavljena iz zvezdic \lstinline{(*)}, deblo pa iz minusov \lstinline{(-)}. Bodi pozoren/a
na presledke.
\begin{lstlisting}
   (*)
  (***)
 (*****)
(*******)
   (-)
   (-)
\end{lstlisting}

\begin{sol}

\textbf{Naloga a)}
\begin{lstlisting}
module Jelka =
struct
    type drevo = {mutable deblo: int ; mutable krosnja : int}
    let ustvari () = {deblo=0; krosnja=0}
    let povejVisino drevo = (drevo.deblo + drevo.krosnja)
    let dolociDeblo drevo v= drevo.deblo <- v
    let dolociKrosnjo drevo v= drevo.krosnja <- v
end
\end{lstlisting}
\textbf{Naloga b)}
\begin{lstlisting}
   (*)
  (***)
 (*****)
(*******)
   (-)
   (-)
\end{lstlisting}
\textbf{Resitev:}
\begin{lstlisting}
    (*deblo vrne string ki predstavlja 
    rezino debla v jelki s polmerom krosnje i*)
    let deblo i =
        if i<0 then failwith "Parameter mora biti ne-negativen!" else
        for k = 1 to i do 
            print_string " "
        done; print_string "(-)\n"
        
    (*krosnja vrne j-to vrstico krosnje s polmerom i*)
    let krosnja i j =
        if i<0 || j<1 then failwith "Napacni parametri krosnje!" else
        for k = 1 to i-j+1 do  (*# presledkov*)
            print_string " "
        done; print_string "(";
        for k = 1 to 2*j-1 do  (* # zvezdic *)
            print_string "*"
        done; print_string ")\n"
        
    let izris d = 
        for j = 1 to d.krosnja do
            krosnja (d.krosnja-1) j
        done;
        for j = 1 to d.deblo do
            deblo (d.krosnja-1)
        done
\end{lstlisting}

\end{sol}

\end{ex}



\begin{ex}
Potrebujemo modul za delo s podatkovno strukturo, ki hrani urejeno sekvenco elementov danega parametri\v cnega tipa 'a v nara\v s\v cajo\v cem vrstnem redu. Modul bomo imenovali \lstinline{Usekvenca}. Tip podatkovne strukture, ki predstavlja urejeno sekvenco imenuj \lstinline{Usekvenca.t}.

Ob kreiranju podatkovne strukture tipa Usekvenca.t podamo kot parameter funkcijo primerjaj : 'a -> 'a -> int, ki vrne -1 v primeru da je prvi parameter manj\v si od drugega, 0 v primeru, da sta parametra enaka in 1 v primeru, da je drugi parameter ve\v cji od prvega. Pozor, funkcijo primerjaj si mora modul zapomniti, ker jo potrebuje pri
dodajanju, brisanju in iskanju elementov.

Definiraj naslednje funkcije modula: 
\begin{lstlisting}
 kreiraj : ('a -> 'a -> int) -> Usekvenca.t 
 dodaj : Usekvenca.t -> 'a -> unit 
 izbrisi : Usekvenca.t -> 'a -> unit   
 min : Usekvenca.t -> 'a 
 max : Usekvenca.t -> 'a 
\end{lstlisting}

Implementiraj funkcijo kreiraj, eno izmed funkcij dodaj in izbri\v si ter
eno izmed funkcij min ali max.
\end{ex}



\begin{ex}
Implementirati moramo enostaven kalkulator za ra\v cunanje s celimi
\v stevili, ki uporablja poljsko notacijo. Kalkulator deluje na naslednji
na\v cin:

V primeru da vnesemo \v stevilo, ga da na vrh delovnega sklada, in \v ce
vtipkamo operacijo (plus, minus, deli in mno\v zi) vzame! zadnja dva
operanda iz sklada, izvede \v zeljeno operacijo in rezultat postavi na
vrh sklada.

Primer izvajanja kalkulatorja: 
\begin{lstlisting}
> 1 
1 
> 2 
2 
> plus 
3 
> 2 
2 
> minus 
1 
> 
\end{lstlisting}

a) Definiraj modul Poljski, ki implementira poljski kalkulator. Modul
naj vsebuje naslednje funkcije:

\begin{lstlisting}
(*inicializacija: *)
kreiraj: unit -> Poljski.t      
(*vnese stevilo na sklad in ga vrne:*)
vnos: Poljski.t -> int -> int   
(*sesteje vrhnja dva elem iz sklada in vrne rezultat:*)
plus: Poljski.t -> int    
(*odsteje:  *)
minus: Poljski.t -> int         
(*zmnozi:  *)
mnozi: Poljski.t -> int         
(*deli:  *)
deli: Poljski.t -> int          
\end{lstlisting}

b) Kako bi posplo\v sili kalkulator, da bi znal delati s poljubnim tipom \v stevil npr. tudi z realnimi \v stevili? Skiciraj na kratko re\v sitev. 

c) Dodatna naloga: implementacija b)  

\begin{sol}

\textbf{Resitev:}
\begin{lstlisting}
module Poljski = 
struct
    type t = {mutable sez:int list}
    (*inicializacija: *)
    let kreiraj () = {sez=[]}
    (*vnese stevilo na sklad in ga vrne:*)
    let vnos s i = s.sez <- i::s.sez; i
    (*sesteje vrhnja dva elem iz sklada in vrne rezultat:*)
    let plus s = match s.sez with
        | a::b::rep -> s.sez<- rep; a+b
        | _ -> failwith "Na skladu je premalo stevil!"
    (*odsteje:  *)
    let minus s = match s.sez with
        | a::b::rep -> s.sez<- rep; a-b
        | _ -> failwith "Na skladu je premalo stevil!"
    (*zmnozi:  *)
    let mnozi s = match s.sez with
        | a::b::rep -> s.sez<- rep; a*b
        | _ -> failwith "Na skladu je premalo stevil!"
    (*deli:  *)
    let deli s = match s.sez with
        | a::b::rep -> s.sez<- rep; a/b
        | _ -> failwith "Na skladu je premalo stevil!"
end
\end{lstlisting}
\textbf{Re\v sitev s poljubnim tipom ki podpira osnovne operacije:}
\begin{lstlisting}
module type BASICALGEBRA = 
sig
    type aType
    val ( +++ ) : aType -> aType -> aType
    val ( --- ) : aType -> aType -> aType
    val ( *** ) : aType -> aType -> aType
    val ( /// ) : aType -> aType -> aType
end

module MakePoljski = functor (Algebra:BASICALGEBRA) ->
struct
    open Algebra
    type t = {mutable sez : aType list}
    let kreiraj () = {sez=[]}
    let vnos s i = s.sez <- i::s.sez; i
    let plus s = match s.sez with
        | a::b::rep -> s.sez<- rep; a+++b
        | _ -> failwith "Na skladu je premalo stevil!"
    let minus s = match s.sez with
        | a::b::rep -> s.sez<- rep; a---b
        | _ -> failwith "Na skladu je premalo stevil!"
    let mnozi s = match s.sez with
        | a::b::rep -> s.sez<- rep; a***b
        | _ -> failwith "Na skladu je premalo stevil!"
    let deli s = match s.sez with
        | a::b::rep -> s.sez<- rep; a///b
        | _ -> failwith "Na skladu je premalo stevil!"
end

module Reals =
struct
    type aType = float
    let (+++) a b = a+.b
    let (---) a b = a-.b
    let ( *** ) a b = a*.b
    let (///) a b = a/.b
end

module RealPoljski = MakePoljski(Reals);;
open RealPoljski;;
let ss=kreiraj ();;
vnos ss 3.1;;
vnos ss 3.14;;
vnos ss 3.141;;
vnos ss 3.1415;;
plus ss;;
ss;;
\end{lstlisting}

\end{sol}

\end{ex}



\begin{ex}
Napisati \v zelimo modul Slika za delo s slikami, ki so definirane z naslednjim parametri\v cnim tipom:

\begin{lstlisting}
type 'a slika = { mutable x: int; mutable y: int;
					mutable p: 'a array array };; 
\end{lstlisting}

Slika je torej dvodimenzionalno polje tipa 'a array array. Posamezna to\v cka slike je predstavljena kot vrednost tipa 'a---uporabljamo lahko razli\v cne tipe za predstavitev ene to\v cke slike.

Napi\v si naslednje funkcije modula:

- kreiraj: kreira novo sliko z danimi parametri,

- zrcali\_x: zrcali sliko po x osi in

- zrcali\_y: zrcali sliko po y osi.

Definiraj vmesnik modula, ki omogo\v ca dostop do predstavljenih funkcij.
\end{ex}



\begin{ex}
Definiraj modul za obdelovo dvodimenzionalnih slik Slika. 

a) Slika je predstavljena s trojicami, ki vsebujejo x in y koordinati ter barvo. Vse tri vrednosti so predstavljene s celimi \v stevili.   

Definirati je potrebno tip Slika.tip, s katero predstavimo sliko.

Definiraj funkcijo Slika.kreiraj s katero kreiramo novo sliko. Sam dolo\v ci parametre in rezultat funkcije Slika.kreiraj. 

b) Vzorce lahko predstavimo z manj\v simi slikami. To\v cka na sliki se ujema s to\v cko vzorca, \v ce se ujemata v barvah. Napi\v si funkcijo 
\begin{lstlisting}
Slika.ujemanje : 'a Slika.tip -> 'a Slika.tip -> (init*int) -> bool, 
\end{lstlisting}

ki za dano sliko (1.parameter) in vzorec (2.parameter) ter koordinati (x,y) (3.parameter) vrne true v primeru, da se vzorec ujema s sliko na koordinatah (x,y) in false sicer.

c) (dodatna naloga) Recimo, da bi \v zeleli definirati fleksibilen modul Slika, ki zna delati z razli\v cnimi predstavitvami barv. 

Za delo z barvami si definiramo majhen modul \lstinline{Barva}, ki vsebuje definicijo tipa \lstinline{Barva.tip} in operacijo 
\begin{lstlisting}
Barva.enakost : Barva.tip -> Barva.tip -> bool, 
\end{lstlisting}
ki pove ali sta dve barvi enaki. 
Skiciraj definicijo funktorja Slika in modula Barva.
\end{ex}




\section{Funkcionali}



\begin{ex}
Implementiraj parametri\v cni modul za delo z urejenimi seznami UrejenSeznam. 

Modul UrejenTip, ki slu\v zi kot parameter modulu UrejenSeznam naj bo uporabljen za definicijo tipa elementov seznama ter definicjo funkcije UrejenTip.primerjaj a b, ki vrne 0 v primeru a=b, 1 v primeru a>b in -1 v primeru a<b. Funkcija UrejenTip.primerjaj definira urejenost elementov urejenega seznama.
\end{ex}



\begin{ex}
Podan je modul Stack, ki realizira sklad elementov tipa 'a.. 

\begin{lstlisting}
module type Stack = 
sig 
   type 'a t 
   exception Empty 
   val create: unit -> 'a t 
   val push: 'a -> 'a t -> unit 
   val pop: 'a t -> 'a 
end

\end{lstlisting}
Definiraj modul Queue, ki s pomo\v cjo dveh skladov implementira vrsto. Prvi sklad predstavlja za\v cetek vrste in drugi sklad predstavlja konec vrste. Implementiraj funkciji enqueue and dequeue !

Namig: \v ce je kateri izmed skladov prazen potem lahko drugega "obrnemo". 
\end{ex} 



\begin{ex}
Definiraj modul, ki zna delati z enodimenzionalnimi polji poljubnega (!)
tipa. Tip elementa polja je torej spremenljivka tipa.e

\textbf{Funkcije}: kreiranje, spajanje, uni\v cevanje, odvzem.
\end{ex} 
\begin{ex}
Naredi modul "garaza", ki:

a) odpira in zapira vrata gara\v ze z enim samim ukazom (\v ce so vrata
odprta jih zapre, sicer naj jih odpre);

b) parkira (odpelje/pripelje) avtomobile ali motorje, pri \v cemer je
\v stevilo vozil omejeno, motor pa zasede polovico prostora avtomobila;

c) izpi\v se \v stevilo in tip vozil v parkiranih v gara\v zi.
\end{ex}


