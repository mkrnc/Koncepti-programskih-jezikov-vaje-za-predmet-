\begin{Odgovor}{5}
\begin{verbatim}
function n -> n * (n + 1) * (2*n + 1)/6
\end{verbatim}
\end{Odgovor}
\begin{Odgovor}{6}
\begin{verbatim}

let rec vsota_vrste n = match n with
| 1 -> 1.
| _ -> 1./.2.**(float_of_int n) +. (vsota_vrste (n-1))
\end{verbatim}
\end{Odgovor}
\begin{Odgovor}{7}
\begin{verbatim}
let rec fib n = match n with
| 0 -> 1
| 1 -> 1
| _ -> fib (n-2) + fib (n-1);;
\end{verbatim}
\end{Odgovor}
\begin{Odgovor}{8}
\begin{verbatim}
let rec jeVsota (a, b, c) = match b with
| 0 -> if (a = c) then true
else false
| b -> jeVsota ((naslednjik a), b-1, c)
\end{verbatim}
\end{Odgovor}
\begin{Odgovor}{9}
\begin{verbatim}
let vsota (a,b)(c,d) = (a+c, b+d)

let rec pfib (a,b) = match (a,b) with
| (i, j) when i <= 0 && j<=0 -> (1,1)
| (i, 0) -> pfib (i-1, 0)
| (0, j) -> pfib (0, j-1)
| (i, j) -> vsota (pfib (i-1, j-1)) (pfib (i-2, j-2))
\end{verbatim}
\end{Odgovor}
\begin{Odgovor}{11}
\begin{verbatim}
let rec sestej sez = match sez with
| []->0
| hd::tl -> hd+(sestej tl)
\end{verbatim}
\end{Odgovor}
\begin{Odgovor}{12}
\begin{verbatim}
let rec najdi e = function
| [] -> false
| h::t ->if( h == e) then true else najdi e t

let rec unija l1 l2 =
match l1 with
| [] -> l2
| h::t -> if najdi h l2 then unija t l2
else unija t (h::l2)
\end{verbatim}
\end{Odgovor}
\begin{Odgovor}{13}
\begin{verbatim}
let zdruzi sez1 sez2 = sez1 @sez2

(* ali *)

let rec zdruzi sez1 sez2 = match (sez1, sez2) with
| ([], s) -> s
| (t, []) -> t
| (a::b, c::d) -> if a<=c then [a]@ (zdruzi b (c::d))
else [c]@(zdruzi (a::b) d)
\end{verbatim}
\end{Odgovor}
\begin{Odgovor}{14}
\begin{verbatim}
et rec zdruzi (sez1,sez2) = match (sez1,sez2) with
| ([],x) -> x
| (x,[]) -> x
| (g1::[],g2::r2) -> g1::g2::r2
| (g1::r1,g2::[]) -> g1::g2::r1
| (g1::r1,g2::g22::r2) -> g1::g2::g22:: zdruzi (r1,r2);;
\end{verbatim}
\end{Odgovor}
\begin{Odgovor}{15}
\begin{verbatim}
let rec vecjeod sez n = match sez with
| []->[]
| hd::tl -> if(hd>n) then hd::(vecjeod tl n) else (vecjeod tl n)
\end{verbatim}
\end{Odgovor}
\begin{Odgovor}{16}
\begin{verbatim}
let rec seznamnm n m =
if n > m then []
else n :: seznamnm (n+1) m;;
\end{verbatim}
\end{Odgovor}
\begin{Odgovor}{17}
\begin{verbatim}
let palindrom sez =
sez = List.rev sez
\end{verbatim}
\end{Odgovor}
\begin{Odgovor}{18}
\begin{verbatim}
let rec vsotaSodeLihe sez = match sez with
| [] -> (0, 0)
| a::b -> let (l,s) = vsotaSodeLihe b in
if (a mod 2 = 1) then
(l+a, s)
else
(l, s+a)
\end{verbatim}
\end{Odgovor}
\begin{Odgovor}{19}
\begin{verbatim}
let rec podseznam sez1 sez2 = match (sez1, sez2) with
| ([], _) -> true
| (a::b, c::d) when List.length sez1 <= List.length sez2 -> if (a=c) then podseznam b d else false
| _ -> false
\end{verbatim}
\end{Odgovor}
\begin{Odgovor}{21}
\begin{verbatim}
let rec ace1 sez count = match sez with
| [] -> false
| a::b -> if (count = 0 && a = 'a') then ace1 b 1
else if(count = 1 && a = 'c') then ace1 b 2
else if (count = 2 && a = 'e') then true
else ace1 b count

let ace sez = ace1 sez 0
\end{verbatim}
\end{Odgovor}
\begin{Odgovor}{22}
\begin{verbatim}
let rec fja list = match list with
| [] -> []
| a::[] -> [a]
| a::b when a=0 -> a::(fja b)
| a::b::c when a=1 && b=0 ->[a; b]@(fja c)
| a::b::c when a=1 && b=1 && c=[] -> [2]@(fja c)
| a::b::c::d when a=1 && b=1 -> if c=1 then [3]@(fja d)
else [2; c]@(fja d)
\end{verbatim}
\end{Odgovor}
\begin{Odgovor}{26}
\begin{verbatim}
vsota(3) = 3 + vsota(2)
vsota(2) = 2+ vsota(1)
vsota(1) = 1+ vsota (0)
vsota (0) = 0
vsota(1) = 1+0=1
vsota(2) = 2+1=3
vsota(3) = 3+3=6
\end{verbatim}
\end{Odgovor}
\begin{Odgovor}{27}
\begin{verbatim}
fib(4) = fib(3) + fib(2)
fib(3) = fib(2) + fib(1)
fib(2) = fib(1) + fib(0)=1
fib(2) = 1+0 = 1
fib(3) = 1+1 = 2
fib(4) = 2+1 = 3
\end{verbatim}
\end{Odgovor}
\begin{Odgovor}{28}
\begin{verbatim}
sestej [1;2;3] = 1+ sestej [2;3]
sestej [2;3] = 2+ sestej[1]
sestej[1] = 1+ sestej []
sestej[] = 0
sestej[1] = 1+ 0=1
sestej [2;3] = 2+ 1=3
sestej [1;2;3] = 1+ 3=4
\end{verbatim}
\end{Odgovor}
\begin{Odgovor}{36}
\begin{verbatim}
let obrni polje = let len=Array.length polje in
for i=0 to (len/2) do
let temp = polje.(i) in
polje.(i) <- polje.(len-i-1);
polje.(len-i-1) <- temp
done;
polje;;
\end{verbatim}
\end{Odgovor}
\begin{Odgovor}{38}
\begin{verbatim}
let razcepi drevo =
let a = ref[] in
let b = ref[] in
let rec raz dr = match dr with
| Prazno ->(!a,!b)
| Vozliscea(x,y) -> a := !a @ [x]; raz y
| Vozlisceb(x,y) -> b := !b @ [x]; raz y
in
raz drevo;;
\end{verbatim}
\end{Odgovor}
\begin{Odgovor}{40}
\begin{verbatim}
let rec prestej sez geo_objekt= match sez with
| [] -> 0
| hd::tl -> if(hd = geo_objekt)then 1 + (prestej tl geo_objekt) else prestej tl geo_objekt;;
\end{verbatim}
\end{Odgovor}
\begin{Odgovor}{44}
\begin{verbatim}
type vrstaKarte = Kraljica | Kralj | Fant | Punca
type barvaKarte = Srce | Kara | Pik | Kriz
type simplKarta = Vrsta of vrstaKarte*barvaKarte
let seznamKart = [(Punca,Srce);(Kralj,Kriz);(Fant,Pik)]
\end{verbatim}
\end{Odgovor}
\begin{Odgovor}{49}
\begin{verbatim}
let stEnic polje =
let count = Array.make 1 0 in
for i = 0 to Array.length polje - 1 do
if (polje.(i) = 1) then count.(0) <- count.(0) + 1 done; count.(0);;
\end{verbatim}
\end{Odgovor}
\begin{Odgovor}{50}
\begin{verbatim}
let produkt a b =
let polje = Array.make 5 0 in
for i=0 to 4 do
polje.(i) <- a.(i)*b.(i)
done;
polje;;
\end{verbatim}
\end{Odgovor}
\begin{Odgovor}{79}
\begin{verbatim}
let rec cnta sez = match sez with
| [] -> []
| 'a'::'a'::'a'::r -> 3 :: cnta r
| 'a'::'a'::r -> 2 :: cnta r
| 'a'::r -> 1 :: cnta r
| _::r -> 0 :: cnta r;;
\end{verbatim}
\end{Odgovor}
\begin{Odgovor}{99}
\begin{verbatim}
let rec zdruzi list1 list2 = match (list1, list2) with
| ([], []) -> []
| (a, []) -> a
| ([], b) -> b
| (a::b, c::d) -> if (a < c )then
a::(zdruzi b (c::d))
else if (c<a) then
c::(zdruzi (a::b) d)
else
a::(zdruzi b d)
\end{verbatim}
\end{Odgovor}
\begin{Odgovor}{111}
\begin{verbatim}
let rec stevila n m = let x=m-1 in
if n > x then []
else n :: stevila (n+1) m;;

let rec cikli m n = match n with
| 0 -> []
| _ -> (stevila 0 m)@cikli m (n-1)
\end{verbatim}
\end{Odgovor}
