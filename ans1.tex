\begin{Odgovor}{2.1}
\begin{lstlisting}
function n -> n * (n + 1) * (2*n + 1)/6
\end{lstlisting}
\end{Odgovor}
\begin{Odgovor}{2.2}
\begin{lstlisting}
let rec vsota_vrste n = match n with
| 1 -> 1.
| _ -> 1./.2.**(float_of_int (n-1)) +. (vsota_vrste (n-1))
\end{lstlisting}
\end{Odgovor}
\begin{Odgovor}{2.3}
\begin{lstlisting}
let rec fib n = match n with
| 0 -> 1
| 1 -> 1
| _ -> fib (n-2) + fib (n-1);;
\end{lstlisting}
\end{Odgovor}
\begin{Odgovor}{2.4}
\begin{lstlisting}
let rec jeVsota (a, b, c) = match b with
| 0 -> if (a = c) then true
else false
| b -> jeVsota ((naslednjik a), b-1, c)
\end{lstlisting}
\end{Odgovor}
\begin{Odgovor}{2.5}
\begin{lstlisting}
let vsota (a,b)(c,d) = (a+c, b+d)

let rec pfib (a,b) = match (a,b) with
| (i, j) when i <= 0 && j<=0 -> (1,1)
| (i, 0) -> pfib (i-1, 0)
| (0, j) -> pfib (0, j-1)
| (i, j) -> vsota (pfib (i-1, j-1)) (pfib (i-2, j-2))
\end{lstlisting}
\end{Odgovor}
\begin{Odgovor}{2.7}
\begin{lstlisting}
sestej [1;2;3] = 1+ sestej [2;3]
sestej [2;3] = 2+ sestej[1]
sestej[1] = 1+ sestej []
sestej[] = 0
sestej[1] = 1+ 0=1
sestej [2;3] = 2+ 1=3
sestej [1;2;3] = 1+ 3=4
\end{lstlisting}
\end{Odgovor}
\begin{Odgovor}{2.8}
\begin{lstlisting}
let rec sestej sez = match sez with
| []->0
| hd::tl -> hd+(sestej tl)
\end{lstlisting}
\end{Odgovor}
\begin{Odgovor}{2.9}
\begin{lstlisting}
let rec najdi e = function
| [] -> false
| h::t ->if( h == e) then true else najdi e t

let rec unija l1 l2 =
match l1 with
| [] -> l2
| h::t -> if najdi h l2 then unija t l2
else unija t (h::l2)
\end{lstlisting}
\end{Odgovor}
\begin{Odgovor}{2.10}
\begin{lstlisting}
let zdruzi sez1 sez2 = sez1 @sez2

(* ali *)

let rec zdruzi sez1 sez2 = match (sez1, sez2) with
| ([], s) -> s
| (t, []) -> t
| (a::b, c::d) -> if a<=c then [a]@ (zdruzi b (c::d))
else [c]@(zdruzi (a::b) d)
\end{lstlisting}
\end{Odgovor}
\begin{Odgovor}{2.11}
\begin{lstlisting}
et rec zdruzi (sez1,sez2) = match (sez1,sez2) with
| ([],x) -> x
| (x,[]) -> x
| (g1::[],g2::r2) -> g1::g2::r2
| (g1::r1,g2::[]) -> g1::g2::r1
| (g1::r1,g2::g22::r2) -> g1::g2::g22:: zdruzi (r1,r2);;
\end{lstlisting}
\end{Odgovor}
\begin{Odgovor}{2.12}
\begin{lstlisting}
let rec vecjeod sez n = match sez with
| []->[]
| hd::tl -> if(hd>n) then hd::(vecjeod tl n) else (vecjeod tl n)
\end{lstlisting}
\end{Odgovor}
\begin{Odgovor}{2.13}
\begin{lstlisting}
let rec seznamnm n m =
if n > m then []
else n :: seznamnm (n+1) m;;
\end{lstlisting}
\end{Odgovor}
\begin{Odgovor}{2.14}
\begin{lstlisting}
let palindrom sez =
sez = List.rev sez
\end{lstlisting}
\end{Odgovor}
\begin{Odgovor}{2.15}
\begin{lstlisting}
let rec vsotaSodeLihe sez = match sez with
| [] -> (0, 0)
| a::b -> let (l,s) = vsotaSodeLihe b in
if (a mod 2 = 1) then
(l+a, s)
else
(l, s+a)
\end{lstlisting}
\end{Odgovor}
\begin{Odgovor}{2.16}
\begin{lstlisting}
let rec podseznam sez1 sez2 = match (sez1, sez2) with
| ([], _) -> true
| (a::b, c::d) when List.length sez1 <= List.length sez2 -> if (a=c) then podseznam b d else false
| _ -> false
\end{lstlisting}
\end{Odgovor}
\begin{Odgovor}{2.17}
\begin{lstlisting}
let rec cnta sez = match sez with
| [] -> []
| 'a'::'a'::'a'::r -> 3 :: cnta r
| 'a'::'a'::r -> 2 :: cnta r
| 'a'::r -> 1 :: cnta r
| _::r -> 0 :: cnta r;;
\end{lstlisting}
\end{Odgovor}
\begin{Odgovor}{2.19}
\begin{lstlisting}
let rec ace1 sez count = match sez with
| [] -> false
| a::b -> if (count = 0 && a = 'a') then ace1 b 1
else if(count = 1 && a = 'c') then ace1 b 2
else if (count = 2 && a = 'e') then true
else ace1 b count

let ace sez = ace1 sez 0
\end{lstlisting}
\end{Odgovor}
\begin{Odgovor}{2.20}
\begin{lstlisting}
let rec fja list = match list with
| [] -> []
| a::[] -> [a]
| a::b when a=0 -> a::(fja b)
| a::b::c when a=1 && b=0 ->[a; b]@(fja c)
| a::b::c when a=1 && b=1 && c=[] -> [2]@(fja c)
| a::b::c::d when a=1 && b=1 -> if c=1 then [3]@(fja d)
else [2; c]@(fja d)
\end{lstlisting}
\end{Odgovor}
\begin{Odgovor}{2.21}
\begin{lstlisting}
let rec stevila n m = let x=m-1 in
if n > x then []
else n :: stevila (n+1) m;;

let rec cikli m n = match n with
| 0 -> []
| _ -> (stevila 0 m)@cikli m (n-1)
\end{lstlisting}
\end{Odgovor}
\begin{Odgovor}{2.26}
\begin{lstlisting}
  let rec zdruzi list1 list2 = match (list1, list2) with
  | ([], []) -> []
  | (a, []) -> a
  | ([], b) -> b
  | (a::b, c::d) ->
       if (a < c ) then a::(zdruzi b (c::d))
       else if (c<a) then c::(zdruzi (a::b) d)
            else a::(zdruzi b d)
\end{lstlisting}
  
\end{Odgovor}
\begin{Odgovor}{2.41}
\begin{lstlisting}
let rec eval i = match i with
|Stevilo (x,Oper ('+',i1))  -> x + eval i1
|Stevilo (x,Oper ('-',i1))  -> x - eval i1
|Stevilo (x,Nil) -> x
|_ -> 0;;
\end{lstlisting}
\end{Odgovor}
\begin{Odgovor}{2.42}
Naloga (i):
\begin{lstlisting}
let rec prestej dr = match dr with
| List x -> x
| Drevo (a,b,c) -> prestej a + prestej c
\end{lstlisting}
Naloga (ii):
\begin{lstlisting}
let rec oznaci dr = match dr with
| List a -> dr
| Drevo (d,f,g) -> Drevo (oznaci d, prestej dr ,oznaci g)
\end{lstlisting}
\end{Odgovor}
\begin{Odgovor}{2.56}
\begin{lstlisting}
let obrni polje = let len=Array.length polje in
for i=0 to (len/2) do
let temp = polje.(i) in
polje.(i) <- polje.(len-i-1);
polje.(len-i-1) <- temp
done;
polje;;
\end{lstlisting}
\end{Odgovor}
\begin{Odgovor}{2.58}
\begin{lstlisting}
let razcepi drevo =
let a = ref[] in
let b = ref[] in
let rec raz dr = match dr with
| Prazno ->(!a,!b)
| Vozliscea(x,y) -> a := !a @ [x]; raz y
| Vozlisceb(x,y) -> b := !b @ [x]; raz y
in
raz drevo;;
\end{lstlisting}
\end{Odgovor}
\begin{Odgovor}{2.71}
\begin{lstlisting}
let rec prestej sez geo_objekt= match sez with
| [] -> 0
| hd::tl -> if(hd = geo_objekt)then 1 + (prestej tl geo_objekt) else prestej tl geo_objekt;;
\end{lstlisting}
\end{Odgovor}
\begin{Odgovor}{2.75}
\begin{lstlisting}
type vrstaKarte = Kraljica | Kralj | Fant | Punca
type barvaKarte = Srce | Kara | Pik | Kriz
type simplKarta = Vrsta of vrstaKarte*barvaKarte
let seznamKart = [(Punca,Srce);(Kralj,Kriz);(Fant,Pik)]
\end{lstlisting}
\end{Odgovor}
\begin{Odgovor}{3.5}
\begin{lstlisting}
let stEnic polje =
let count = Array.make 1 0 in
for i = 0 to Array.length polje - 1 do
if (polje.(i) = 1) then count.(0) <- count.(0) + 1 done; count.(0);;
\end{lstlisting}
\end{Odgovor}
\begin{Odgovor}{3.6}
\begin{lstlisting}
let produkt a b =
let polje = Array.make 5 0 in
for i=0 to 4 do
polje.(i) <- a.(i)*b.(i)
done;
polje;;
\end{lstlisting}
\end{Odgovor}
\begin{Odgovor}{3.14}
\begin{lstlisting}
type piksel = {barva:int; intenziteta:int}
type slika = piksel array array

let pika s (i,j) = s.(i).(j)#intetnziteta > 0

let poisci s =
    let ret=ref (0,0) in
    for i = 0 to 100 do
        for j = 0 to 100 do
            if pika s (i,j) then ret := (i,j-2)
    done done; !ret
\end{lstlisting}
 
\end{Odgovor}
\begin{Odgovor}{3.15}

\textbf{Resitev 1. podnaloge:}
\begin{lstlisting}
(* resitev od Kristofa *)
type tekst = {mutable matrika: char array array};;

(*primer instance tipa tekst*)
let a = {matrika=Array.make_matrix 100 1000 ' '};;
for i=0 to 99 do
    for j=0 to 999 do
        a.matrika.(i).(j) <- char_of_int((i+j) mod 256)
    done
done
\end{lstlisting}

\textbf{Resitev 2. podnaloge:}
\begin{lstlisting}
let zamenjaj_navpicno zaslon a b =
    let n=(String.length a) in
    for j=0 to 99 do (*za vsak stolpec*)
        let i=ref 0 in
        while !i < (1000-n) do
            if preveri_ujemanje zaslon.matrika.(j) !i a then
                (zamenjaj zaslon.matrika.(j) !i b;
                i:=!i+n;)
            else
                i:=!i+1
        done
    done;;


(* napisi funkcijo zamenjaj, ki v podanem stolpcu : char array
na i-tem mestu zamenja podniz dolzine n z vhodnim nizom dolzine n*)
(*resitev od Sebastijana*)
let zamenjaj stolpec i niz =
    let pos = ref 0 in
    for j = i to i+(String.length niz)-1 do
        stolpec.(j) <- niz.[!pos] ; pos := !pos + 1
    done;;

(*napisi funkcijo preveri_ujemanje, ki v podanem stolpcu : char array
na i-tem mestu preveri ce se podniz dolzine n ujema z vhodnim nizom
dolzine n*)

(*resitev od Kristofa*)
let preveri_ujemanje stolpec i niz =
    let same = ref true and
    str_index = ref 0 in
    while !same && !str_index < (String.length niz) do
        if (stolpec.(i + (!str_index)) == niz.[!str_index]) then
            str_index := !str_index +1
        else
            same := false
    done;
    !same;;

zamenjaj_navpicno a "nop" "XXX";;
zamenjaj_navpicno a "rst" "XXX";;

\end{lstlisting}

\end{Odgovor}
\begin{Odgovor}{3.29}
\begin{lstlisting}
vsota(3) = 3 + vsota(2)
vsota(2) = 2+ vsota(1)
vsota(1) = 1+ vsota (0)
vsota (0) = 0
vsota(1) = 1+0=1
vsota(2) = 2+1=3
vsota(3) = 3+3=6
\end{lstlisting}
\end{Odgovor}
\begin{Odgovor}{3.30}
\begin{lstlisting}
fib(4) = fib(3) + fib(2)
fib(3) = fib(2) + fib(1)
fib(2) = fib(1) + fib(0)=1
fib(2) = 1+0 = 1
fib(3) = 1+1 = 2
fib(4) = 2+1 = 3
\end{lstlisting}
\end{Odgovor}
\begin{Odgovor}{4.1}
\begin{lstlisting}
class stevila st =
object
    val st=(st:int)
    method vrednost = st
    method sestej a = new integers (st+a)
    method odstej a = new integers (st-a)
    method pomnozi a = new integers (st*a)
    method deljenje a = new integers (st/a)
end

class poz_st st =
object (self)
    inherit stevila st as super
    method sestej a =
        if a<0 then
            super#sestej (abs(a)*100)
        else
            super#sestej a
    method odstej a =
        if a<0 then
            super#odstej (abs(a)*100)
        else
            super#odstej a
    method pomnozi a =
        if a<0 then
            super#pomnozi (abs(a)*100)
        else
            super#pomnozi a
    method deljenje a =
        if a<0 then
            super#deljenje (abs(a)*100)
        else
            super#deljenje a
end
\end{lstlisting}
\end{Odgovor}
\begin{Odgovor}{4.14}
\begin{lstlisting}
(* predpostavimo da imamo
narisi_tocko : int*int -> unit
narisi_premico : int*int -> int*int -> unit
narisi_krog : int*int -> int -> unit
*)

let narisi_tocko a b = ();;
let narisi_premico a b = ();;
let narisi_krog a b = ();;

class virtual geo =
object
    method virtual predstavi : string
    method virtual testiraj : float -> float -> bool
    method virtual narisi : unit
end;;

class tocka x y =
object
    inherit geo
    val x = (x:float)
    val y = (y:float)
    method predstavi = "Tocka s koordinatami ("^(string_of_float x)
        ^","^(string_of_float x^").")
    method testiraj tx ty = x=tx && y=ty
    method narisi = narisi_tocko x y
end;;

let t1 = new tocka 1. 2.;;
t1#predstavi;;
t1#testiraj 1. 2.;;
t1#testiraj 1. 2.2;;

class premica k n = (* f(x)=kx+n*)
object
    inherit geo
    val k = k
    val n = n
    method predstavi = "Premica s predpisom f(x)="^(string_of_float k)
        ^"* x + "^(string_of_float n^".")
    method testiraj tx ty = (ty = k *. tx +. n)
    method narisi = narisi_premico (0.,n) (1.,n+.k)
end;;

let p1 = new premica 2. 0.;;
p1#predstavi;;
p1#testiraj 1. 2.2;;
p1#testiraj 1. 2.;;

class krog sx sy r =  (* krog polmera r s srediscem v (sx,sy)*)
object
    inherit geo
    val sx=sx
    val sy=sy
    val r = r
    method predstavi = "Krog s polmerom r="^(string_of_float r)
        ^" ter srediscem v ("^(string_of_float sx)
        ^","^(string_of_float sy)^")."
    method testiraj tx ty = (tx-.sx)**2. +. (ty-.sy)**2. <= r**2.
    method narisi = narisi_krog (sx,sy) r
end

let k1 = new krog 1. 1. 1.;;
k1#predstavi;;
k1#testiraj 1. (-0.);;
k1#testiraj 1. (-0.000001);;
\end{lstlisting}

\end{Odgovor}
\begin{Odgovor}{4.17}

\begin{lstlisting}
type 'a option =
| None
| Some of 'a;;

class ['a] polje n =
object
    val p = ((Array.make n None) : 'a option array)
    method popravi i a = p.(i) <- (Some a)
    method preberi i = p.(i)
end;;

let p1 = new polje 12;;
p1#preberi 2;;
p1#popravi 2 3.14;;
p1#preberi 2;;

class real_polje n =
object
    inherit [float] polje n
    method preberi i = match p.(i) with
    |None -> Some 0.
    |Some x -> Some (float_of_int (int_of_float x))
end
\end{lstlisting}
\end{Odgovor}
\begin{Odgovor}{5.3}

\begin{lstlisting}
module TAgencija =
struct
    type aranzma ={ mutable destinacija:string;
                    mutable tip_namestitve:string;
                    mutable trajanje:int;
                    mutable cena:int}
    let kreiraj dest tip trajanje cena = {destinacija=dest;
                    tip_namestitve=tip;
                    trajanje=trajanje;
                    cena=cena}
    let ogled a = a.destinacija,a.tip_namestitve, a.trajanje,a.cena
    let popravi_dest a nova_dest = a.destinacija <- nova_dest
    let popravi_t_n a nov_t_n = a.tip_namestitve <- nov_t_n
    let popravi_trajanje a novo_trajanje = a.trajanje <- novo_trajanje
    let popravi_ceno a nova_cena = a.cena <- nova_cena
end;;

module type AGENT =
  sig
    type aranzma
    val kreiraj : string -> string -> int -> int -> aranzma
    val ogled : aranzma -> string * string * int * int
    val popravi_dest : aranzma -> string -> unit
    val popravi_t_n : aranzma -> string -> unit
    val popravi_trajanje : aranzma -> int -> unit
    val popravi_ceno : aranzma -> int -> unit
  end;;

module type STRANKA =
  sig
    type aranzma
    val ogled : aranzma -> string * string * int * int
  end;;

module Agent = (TAgencija:AGENT with type aranzma = TAgencija.aranzma);;
module Stranka = (TAgencija:STRANKA with type aranzma = TAgencija.aranzma);;

\end{lstlisting}

\end{Odgovor}
\begin{Odgovor}{5.4}

\textbf{Naloga a)}
\begin{lstlisting}
module Jelka =
struct
    type drevo = {mutable deblo: int ; mutable krosnja : int}
    let ustvari () = {deblo=0; krosnja=0}
    let povejVisino drevo = (drevo.deblo + drevo.krosnja)
    let dolociDeblo drevo v= drevo.deblo <- v
    let dolociKrosnjo drevo v= drevo.krosnja <- v
end
\end{lstlisting}
\textbf{Naloga b)}
\begin{lstlisting}
   (*)
  (***)
 (*****)
(*******)
   (-)
   (-)
\end{lstlisting}
\textbf{Resitev:}
\begin{lstlisting}
    (*deblo vrne string ki predstavlja
    rezino debla v jelki s polmerom krosnje i*)
    let deblo i =
        if i<0 then failwith "Parameter mora biti ne-negativen!" else
        for k = 1 to i do
            print_string " "
        done; print_string "(-)\n"

    (*krosnja vrne j-to vrstico krosnje s polmerom i*)
    let krosnja i j =
        if i<0 || j<1 then failwith "Napacni parametri krosnje!" else
        for k = 1 to i-j+1 do  (*# presledkov*)
            print_string " "
        done; print_string "(";
        for k = 1 to 2*j-1 do  (* # zvezdic *)
            print_string "*"
        done; print_string ")\n"

    let izris d =
        for j = 1 to d.krosnja do
            krosnja (d.krosnja-1) j
        done;
        for j = 1 to d.deblo do
            deblo (d.krosnja-1)
        done
\end{lstlisting}

\end{Odgovor}
\begin{Odgovor}{5.6}

\textbf{Resitev:}
\begin{lstlisting}
module Poljski =
struct
    type t = {mutable sez:int list}
    (*inicializacija: *)
    let kreiraj () = {sez=[]}
    (*vnese stevilo na sklad in ga vrne:*)
    let vnos s i = s.sez <- i::s.sez; i
    (*sesteje vrhnja dva elem iz sklada in vrne rezultat:*)
    let plus s = match s.sez with
        | a::b::rep -> s.sez<- rep; a+b
        | _ -> failwith "Na skladu je premalo stevil!"
    (*odsteje:  *)
    let minus s = match s.sez with
        | a::b::rep -> s.sez<- rep; a-b
        | _ -> failwith "Na skladu je premalo stevil!"
    (*zmnozi:  *)
    let mnozi s = match s.sez with
        | a::b::rep -> s.sez<- rep; a*b
        | _ -> failwith "Na skladu je premalo stevil!"
    (*deli:  *)
    let deli s = match s.sez with
        | a::b::rep -> s.sez<- rep; a/b
        | _ -> failwith "Na skladu je premalo stevil!"
end
\end{lstlisting}
\textbf{Re\v sitev s poljubnim tipom ki podpira osnovne operacije:}
\begin{lstlisting}
module type BASICALGEBRA =
sig
    type aType
    val ( +++ ) : aType -> aType -> aType
    val ( --- ) : aType -> aType -> aType
    val ( *** ) : aType -> aType -> aType
    val ( /// ) : aType -> aType -> aType
end

module MakePoljski = functor (Algebra:BASICALGEBRA) ->
struct
    open Algebra
    type t = {mutable sez : aType list}
    let kreiraj () = {sez=[]}
    let vnos s i = s.sez <- i::s.sez; i
    let plus s = match s.sez with
        | a::b::rep -> s.sez<- rep; a+++b
        | _ -> failwith "Na skladu je premalo stevil!"
    let minus s = match s.sez with
        | a::b::rep -> s.sez<- rep; a---b
        | _ -> failwith "Na skladu je premalo stevil!"
    let mnozi s = match s.sez with
        | a::b::rep -> s.sez<- rep; a***b
        | _ -> failwith "Na skladu je premalo stevil!"
    let deli s = match s.sez with
        | a::b::rep -> s.sez<- rep; a///b
        | _ -> failwith "Na skladu je premalo stevil!"
end

module Reals =
struct
    type aType = float
    let (+++) a b = a+.b
    let (---) a b = a-.b
    let ( *** ) a b = a*.b
    let (///) a b = a/.b
end

module RealPoljski = MakePoljski(Reals);;
open RealPoljski;;
let ss=kreiraj ();;
vnos ss 3.1;;
vnos ss 3.14;;
vnos ss 3.141;;
vnos ss 3.1415;;
plus ss;;
ss;;
\end{lstlisting}

\end{Odgovor}
