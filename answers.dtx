%    \iffalse meta-comment
%
% Originally developed by Mike Piff 
% Copyright (C) 1990,1994-1996 by Mike Piff   
% Copyright (C) 2009-2010,2014 by
%   Joseph Wright <joseph.wright@morningstar2.co.uk>
%
% This file may be distributed and/or modified under the conditions of
% the LaTeX Project Public License (LPPL), either version 1.3c of
% this license or (at your option) any later version.  The latest
% version of this license is in the file:
%
%   http://www.latex-project.org/lppl.txt
%
% This work is "maintained" (as per LPPL maintenance status) by
%   Joseph Wright.
%
%    \fi
%
%    \CheckSum{267}
%
% \changes{2.0}{1994/03/25}{First version for LaTeX2e}
% \changes{2.01}{1994/03/30}{Whoops! |ProvidesPackage{answers}|,
%           not |ProvidesClass{Answers}|!!}
% \changes{2.02}{1995/08/31}{Removed .drv file from installation}
% \changes{2.03}{1995/08/31}{Allowed a file handle to open
%           many files successively, using an optional parameter.}
% \changes{2.04}{1995/09/08}{Option nosolutionfiles added to allow
%           solutions to optionally follow problems for tutor's use.
%           Several minor changes to aid flexibility}
% \changes{2.05}{1995/09/11}{Better documentation of features added}
% \changes{2.06}{1996/01/05}{More documentation of features added,
%           as suggested by David Epstein.}
% \changes{2.07}{1996/01/08}{Conditionals to check if files were open
%           were not changed globally. (Spotted by David Epstein.)}
% \changes{2.08}{1996/01/11}{More examples added.}
% \changes{2.09}{1996/07/10}{Bug spotted by David Epstein;
%           I had missed out the definition of the boolean that
%           checks if a file is open!  Ah well!}
% \changes{2.12}{2009/09/16}{License change to LPPL}
% \changes{2.12}{2009/09/16}{New maintainer}
% \changes{2.13}{2010/10/11}{Make hyperlinks to solutions work correctly}
% \changes{2.14}{2014/08/08}{Use a protected write for creating output files}
% \changes{2.15}{2014/08/14}{Fix bug in last update}
% \changes{2.16}{2014/08/24}{Fix bug in last update}
%
%
%    \title{Production of solution sheets in \LaTeXe}
%    \author{Mike Piff \and Joseph Wright\thanks{joseph.wright@morningstar2.co.uk}}
%
%    \maketitle
%    \tableofcontents
%
%    \section{Introduction}
%
%    This package is a modification of the author's previous style option
%    |answers|, which has been in use for a few years,
%    and was based upon the
%    \TeX book idea of binding solutions to exercises. I have taken the
%    opportunity with this revision to alter the format of the solutions,
%    so that they are now presented as \LaTeX\ environments rather than
%    being started with a command and ended with the end of the surrounding
%    environment, a wholly un-\LaTeX y way of doing things!
%
%    The other main change is that several file
%    handles are allowed to be active
%    at once. This allows some solutions in a book (for instance) to go
%    to the appendices, and some to go to a separate file, to be printed and
%    handed to the students as the course progresses.
%    Moreover, the actual physical files opened with each file handle
%    can now be varied in the same job, allowing many different files
%    to be created according to the same format.  Thus, for instance,
%    each chapter of a book could create its own solution file, allowing
%    the user to use |\include| on both chapters and solutions.
%
%    Finally, any number of solution-types may now be bound to any file, not
%    just the two old ones, solution and hint.   The format of each solution
%    type is under the complete control of the user.
%
%    \section{The documentation driver file}
%
%    This is the driver file that produces this documentation.
%    We use the document class provided by the \LaTeXe\ distribution
%    for producing the documentation.
%    \begin{macrocode}
%<*driver>
\documentclass{ltxdoc}
\RecordChanges
\begin{document}
  \DocInput{answers.dtx}
  \PrintIndex
  \PrintChanges
\end{document}
%</driver>
%    \end{macrocode}
%    
%    \section{User interface}
%
%    The package needs to be included with the command
%    \begin{verbatim}
%             \usepackage[nosolutionfiles]{answers}
%    \end{verbatim}
%    If the optional argument is given, solutions appear at that point
%    in the text, rather than being written to external files.
%    This allows a demonstrator's version to be produced.
%
%    \DescribeMacro{\Newassociation}
%    After that, there should be several declarations of the form
%    \begin{verbatim}
%        \Newassociation{xxx}{yyy}{zzz}
%    \end{verbatim}
%    where |xxx| is an environment in the document, and |yyy| is an
%    environment which will surround the contents of |xxx| when it is
%    written to symbolic file handle |zzz|.
%    The names |xxx|, |yyy| and |zzz| should consist of letters
%    only, not numbers, punctuation or spaces.
%
%    \DescribeMacro{\solutionextension}
%    By default, output will go to |zzz.tex| if |zzz| is open.
%    The command |\solutionextension| can be redefined to change |tex| to
%    some other extension.  Alternatively, the output filename can be
%    changed as an optional parameter to |\Opensolutionfile|, and each
%    |\Opensolutionfile| on the same handle can use a different physical
%    file.  By default, |\solutionpoint| is added before |\solutionextension|.
%	  Redefine it to remove it. (It has the obvious default value of a
%    period.
%
%    \DescribeMacro{\Opensolutionfile}
%    \DescribeMacro{\Closesolutionfile}
%    At some point the user types
%    \begin{verbatim}
%        \Opensolutionfile{zzz}
%        ...
%        \Closesolutionfile{zzz}
%    \end{verbatim}
%    to create a file of solutions written by environments |xxx| to
%    environments |yyy|.
%    If this construction is used several times, then several files
%    of solutions will be created.
%    The user may wish these files to have different names.
%    If the form |\Opensolutionfile{zzz}[www]|, then |www.tex| is used as
%    actual file output name rather than |zzz.tex|.   This allows file
%    handle |zzz| to create many files |www.tex|, say one for each
%    chapter of a book, or one for each problem sheet.  These could then
%    be processed using |\include| commands.  The same value of
%    |\solutionextension| is used for the optional argument as for the
%    main argument.
%    The name |www| should follow the usual file naming conventions.
%
%    \DescribeMacro{\Writetofile}
%    In addition, material can be written directly to a file by means of
%    |\Writetofile|.
%    Its first argument is the file handle |zzz| and its second is
%    the line of text to be written.
%    It is  most important to remember that any control words
%    in the line to be written should be preceded by |\protect|,
%    otherwise the primitive \TeX\ |\write| command will expand them.
%    Also, as the argument is read in \TeX' usual way before being
%    written, any trailing spaces after a control word will disappear
%    unless precautions are taken.  Thus, to write |\xx yyy| to the file,
%    the user can type |\protect\xx\space yyy|.
%
%    \DescribeEnv{Filesave}
%    Alternatively, a block of text can be saved to file handle
%    |zzz| by means of
%    \begin{verbatim}
%        \begin{Filesave}{zzz}
%           ....
%        \end{Filesave}
%    \end{verbatim}
%    around it once, |zzz| has been opened.
%    The restrictions that apply to |\Writetofile| above do not apply
%    to this environment.
%
%    \DescribeMacro{\Readsolutionfile}
%    One of the generated files can be read using
%    \begin{verbatim}
%        \Readsolutionfile{zzz}
%    \end{verbatim}
%    provided the file has not been closed and re-opened.
%    Alternatively, simply |\input| or |\include| it if preferred.
%
%     None of the file operations should have any effect if the file
%     handle |zzz| has not been opened, or if |nosolutionfiles| is specified.
%
%    \section{A simple example}
%
%    Here is a straightforward example to illustrate how these macros
%    are used.
%    \begin{macrocode}
%<*ex1>
\documentclass[12pt,a4paper]{article}
\usepackage{answers}
\Newassociation{sol}{Solution}{ans}
\newtheorem{ex}{Exercise}
\begin{document}
\Opensolutionfile{ans}[ans1]
\section{Problems}
\begin{ex}
   First exercise
   \begin{sol}
      First solution.
   \end{sol}
\end{ex}
\begin{ex}
   Second exercise
   \begin{sol}
      Second solution.
   \end{sol}
\end{ex}
\Closesolutionfile{ans}
\section{Solutions}
\begin{Odgovor}{2.1}
\begin{lstlisting}
function n -> n * (n + 1) * (2*n + 1)/6
\end{lstlisting}
\end{Odgovor}
\begin{Odgovor}{2.2}
\begin{lstlisting}
let rec vsota_vrste n = match n with
| 1 -> 1.
| _ -> 1./.2.**(float_of_int (n-1)) +. (vsota_vrste (n-1))
\end{lstlisting}
\end{Odgovor}
\begin{Odgovor}{2.3}
\begin{lstlisting}
let rec fib n = match n with
| 0 -> 1
| 1 -> 1
| _ -> fib (n-2) + fib (n-1);;
\end{lstlisting}
\end{Odgovor}
\begin{Odgovor}{2.4}
\begin{lstlisting}
let rec jeVsota (a, b, c) = match b with
| 0 -> if (a = c) then true
else false
| b -> jeVsota ((naslednjik a), b-1, c)
\end{lstlisting}
\end{Odgovor}
\begin{Odgovor}{2.5}
\begin{lstlisting}
let vsota (a,b)(c,d) = (a+c, b+d)

let rec pfib (a,b) = match (a,b) with
| (i, j) when i <= 0 && j<=0 -> (1,1)
| (i, 0) -> pfib (i-1, 0)
| (0, j) -> pfib (0, j-1)
| (i, j) -> vsota (pfib (i-1, j-1)) (pfib (i-2, j-2))
\end{lstlisting}
\end{Odgovor}
\begin{Odgovor}{2.7}
\begin{lstlisting}
sestej [1;2;3] = 1+ sestej [2;3]
sestej [2;3] = 2+ sestej[1]
sestej[1] = 1+ sestej []
sestej[] = 0
sestej[1] = 1+ 0=1
sestej [2;3] = 2+ 1=3
sestej [1;2;3] = 1+ 3=4
\end{lstlisting}
\end{Odgovor}
\begin{Odgovor}{2.8}
\begin{lstlisting}
let rec sestej sez = match sez with
| []->0
| hd::tl -> hd+(sestej tl)
\end{lstlisting}
\end{Odgovor}
\begin{Odgovor}{2.9}
\begin{lstlisting}
let rec najdi e = function
| [] -> false
| h::t ->if( h == e) then true else najdi e t

let rec unija l1 l2 =
match l1 with
| [] -> l2
| h::t -> if najdi h l2 then unija t l2
else unija t (h::l2)
\end{lstlisting}
\end{Odgovor}
\begin{Odgovor}{2.10}
\begin{lstlisting}
let zdruzi sez1 sez2 = sez1 @sez2

(* ali *)

let rec zdruzi sez1 sez2 = match (sez1, sez2) with
| ([], s) -> s
| (t, []) -> t
| (a::b, c::d) -> if a<=c then [a]@ (zdruzi b (c::d))
else [c]@(zdruzi (a::b) d)
\end{lstlisting}
\end{Odgovor}
\begin{Odgovor}{2.11}
\begin{lstlisting}
et rec zdruzi (sez1,sez2) = match (sez1,sez2) with
| ([],x) -> x
| (x,[]) -> x
| (g1::[],g2::r2) -> g1::g2::r2
| (g1::r1,g2::[]) -> g1::g2::r1
| (g1::r1,g2::g22::r2) -> g1::g2::g22:: zdruzi (r1,r2);;
\end{lstlisting}
\end{Odgovor}
\begin{Odgovor}{2.12}
\begin{lstlisting}
let rec vecjeod sez n = match sez with
| []->[]
| hd::tl -> if(hd>n) then hd::(vecjeod tl n) else (vecjeod tl n)
\end{lstlisting}
\end{Odgovor}
\begin{Odgovor}{2.13}
\begin{lstlisting}
let rec seznamnm n m =
if n > m then []
else n :: seznamnm (n+1) m;;
\end{lstlisting}
\end{Odgovor}
\begin{Odgovor}{2.14}
\begin{lstlisting}
let palindrom sez =
sez = List.rev sez
\end{lstlisting}
\end{Odgovor}
\begin{Odgovor}{2.15}
\begin{lstlisting}
let rec vsotaSodeLihe sez = match sez with
| [] -> (0, 0)
| a::b -> let (l,s) = vsotaSodeLihe b in
if (a mod 2 = 1) then
(l+a, s)
else
(l, s+a)
\end{lstlisting}
\end{Odgovor}
\begin{Odgovor}{2.16}
\begin{lstlisting}
let rec podseznam sez1 sez2 = match (sez1, sez2) with
| ([], _) -> true
| (a::b, c::d) when List.length sez1 <= List.length sez2 -> if (a=c) then podseznam b d else false
| _ -> false
\end{lstlisting}
\end{Odgovor}
\begin{Odgovor}{2.17}
\begin{lstlisting}
let rec cnta sez = match sez with
| [] -> []
| 'a'::'a'::'a'::r -> 3 :: cnta r
| 'a'::'a'::r -> 2 :: cnta r
| 'a'::r -> 1 :: cnta r
| _::r -> 0 :: cnta r;;
\end{lstlisting}
\end{Odgovor}
\begin{Odgovor}{2.19}
\begin{lstlisting}
let rec ace1 sez count = match sez with
| [] -> false
| a::b -> if (count = 0 && a = 'a') then ace1 b 1
else if(count = 1 && a = 'c') then ace1 b 2
else if (count = 2 && a = 'e') then true
else ace1 b count

let ace sez = ace1 sez 0
\end{lstlisting}
\end{Odgovor}
\begin{Odgovor}{2.20}
\begin{lstlisting}
let rec fja list = match list with
| [] -> []
| a::[] -> [a]
| a::b when a=0 -> a::(fja b)
| a::b::c when a=1 && b=0 ->[a; b]@(fja c)
| a::b::c when a=1 && b=1 && c=[] -> [2]@(fja c)
| a::b::c::d when a=1 && b=1 -> if c=1 then [3]@(fja d)
else [2; c]@(fja d)
\end{lstlisting}
\end{Odgovor}
\begin{Odgovor}{2.21}
\begin{lstlisting}
let rec stevila n m = let x=m-1 in
if n > x then []
else n :: stevila (n+1) m;;

let rec cikli m n = match n with
| 0 -> []
| _ -> (stevila 0 m)@cikli m (n-1)
\end{lstlisting}
\end{Odgovor}
\begin{Odgovor}{2.26}
\begin{lstlisting}
  let rec zdruzi list1 list2 = match (list1, list2) with
  | ([], []) -> []
  | (a, []) -> a
  | ([], b) -> b
  | (a::b, c::d) ->
       if (a < c ) then a::(zdruzi b (c::d))
       else if (c<a) then c::(zdruzi (a::b) d)
            else a::(zdruzi b d)
\end{lstlisting}
  
\end{Odgovor}
\begin{Odgovor}{2.41}
\begin{lstlisting}
let rec eval i = match i with
|Stevilo (x,Oper ('+',i1))  -> x + eval i1
|Stevilo (x,Oper ('-',i1))  -> x - eval i1
|Stevilo (x,Nil) -> x
|_ -> 0;;
\end{lstlisting}
\end{Odgovor}
\begin{Odgovor}{2.42}
Naloga (i):
\begin{lstlisting}
let rec prestej dr = match dr with
| List x -> x
| Drevo (a,b,c) -> prestej a + prestej c
\end{lstlisting}
Naloga (ii):
\begin{lstlisting}
let rec oznaci dr = match dr with
| List a -> dr
| Drevo (d,f,g) -> Drevo (oznaci d, prestej dr ,oznaci g)
\end{lstlisting}
\end{Odgovor}
\begin{Odgovor}{2.56}
\begin{lstlisting}
let obrni polje = let len=Array.length polje in
for i=0 to (len/2) do
let temp = polje.(i) in
polje.(i) <- polje.(len-i-1);
polje.(len-i-1) <- temp
done;
polje;;
\end{lstlisting}
\end{Odgovor}
\begin{Odgovor}{2.58}
\begin{lstlisting}
let razcepi drevo =
let a = ref[] in
let b = ref[] in
let rec raz dr = match dr with
| Prazno ->(!a,!b)
| Vozliscea(x,y) -> a := !a @ [x]; raz y
| Vozlisceb(x,y) -> b := !b @ [x]; raz y
in
raz drevo;;
\end{lstlisting}
\end{Odgovor}
\begin{Odgovor}{2.71}
\begin{lstlisting}
let rec prestej sez geo_objekt= match sez with
| [] -> 0
| hd::tl -> if(hd = geo_objekt)then 1 + (prestej tl geo_objekt) else prestej tl geo_objekt;;
\end{lstlisting}
\end{Odgovor}
\begin{Odgovor}{2.75}
\begin{lstlisting}
type vrstaKarte = Kraljica | Kralj | Fant | Punca
type barvaKarte = Srce | Kara | Pik | Kriz
type simplKarta = Vrsta of vrstaKarte*barvaKarte
let seznamKart = [(Punca,Srce);(Kralj,Kriz);(Fant,Pik)]
\end{lstlisting}
\end{Odgovor}
\begin{Odgovor}{3.5}
\begin{lstlisting}
let stEnic polje =
let count = Array.make 1 0 in
for i = 0 to Array.length polje - 1 do
if (polje.(i) = 1) then count.(0) <- count.(0) + 1 done; count.(0);;
\end{lstlisting}
\end{Odgovor}
\begin{Odgovor}{3.6}
\begin{lstlisting}
let produkt a b =
let polje = Array.make 5 0 in
for i=0 to 4 do
polje.(i) <- a.(i)*b.(i)
done;
polje;;
\end{lstlisting}
\end{Odgovor}
\begin{Odgovor}{3.14}
\begin{lstlisting}
type piksel = {barva:int; intenziteta:int}
type slika = piksel array array

let pika s (i,j) = s.(i).(j)#intetnziteta > 0

let poisci s =
    let ret=ref (0,0) in
    for i = 0 to 100 do
        for j = 0 to 100 do
            if pika s (i,j) then ret := (i,j-2)
    done done; !ret
\end{lstlisting}
 
\end{Odgovor}
\begin{Odgovor}{3.15}

\textbf{Resitev 1. podnaloge:}
\begin{lstlisting}
(* resitev od Kristofa *)
type tekst = {mutable matrika: char array array};;

(*primer instance tipa tekst*)
let a = {matrika=Array.make_matrix 100 1000 ' '};;
for i=0 to 99 do
    for j=0 to 999 do
        a.matrika.(i).(j) <- char_of_int((i+j) mod 256)
    done
done
\end{lstlisting}

\textbf{Resitev 2. podnaloge:}
\begin{lstlisting}
let zamenjaj_navpicno zaslon a b =
    let n=(String.length a) in
    for j=0 to 99 do (*za vsak stolpec*)
        let i=ref 0 in
        while !i < (1000-n) do
            if preveri_ujemanje zaslon.matrika.(j) !i a then
                (zamenjaj zaslon.matrika.(j) !i b;
                i:=!i+n;)
            else
                i:=!i+1
        done
    done;;


(* napisi funkcijo zamenjaj, ki v podanem stolpcu : char array
na i-tem mestu zamenja podniz dolzine n z vhodnim nizom dolzine n*)
(*resitev od Sebastijana*)
let zamenjaj stolpec i niz =
    let pos = ref 0 in
    for j = i to i+(String.length niz)-1 do
        stolpec.(j) <- niz.[!pos] ; pos := !pos + 1
    done;;

(*napisi funkcijo preveri_ujemanje, ki v podanem stolpcu : char array
na i-tem mestu preveri ce se podniz dolzine n ujema z vhodnim nizom
dolzine n*)

(*resitev od Kristofa*)
let preveri_ujemanje stolpec i niz =
    let same = ref true and
    str_index = ref 0 in
    while !same && !str_index < (String.length niz) do
        if (stolpec.(i + (!str_index)) == niz.[!str_index]) then
            str_index := !str_index +1
        else
            same := false
    done;
    !same;;

zamenjaj_navpicno a "nop" "XXX";;
zamenjaj_navpicno a "rst" "XXX";;

\end{lstlisting}

\end{Odgovor}
\begin{Odgovor}{3.29}
\begin{lstlisting}
vsota(3) = 3 + vsota(2)
vsota(2) = 2+ vsota(1)
vsota(1) = 1+ vsota (0)
vsota (0) = 0
vsota(1) = 1+0=1
vsota(2) = 2+1=3
vsota(3) = 3+3=6
\end{lstlisting}
\end{Odgovor}
\begin{Odgovor}{3.30}
\begin{lstlisting}
fib(4) = fib(3) + fib(2)
fib(3) = fib(2) + fib(1)
fib(2) = fib(1) + fib(0)=1
fib(2) = 1+0 = 1
fib(3) = 1+1 = 2
fib(4) = 2+1 = 3
\end{lstlisting}
\end{Odgovor}
\begin{Odgovor}{4.1}
\begin{lstlisting}
class stevila st =
object
    val st=(st:int)
    method vrednost = st
    method sestej a = new integers (st+a)
    method odstej a = new integers (st-a)
    method pomnozi a = new integers (st*a)
    method deljenje a = new integers (st/a)
end

class poz_st st =
object (self)
    inherit stevila st as super
    method sestej a =
        if a<0 then
            super#sestej (abs(a)*100)
        else
            super#sestej a
    method odstej a =
        if a<0 then
            super#odstej (abs(a)*100)
        else
            super#odstej a
    method pomnozi a =
        if a<0 then
            super#pomnozi (abs(a)*100)
        else
            super#pomnozi a
    method deljenje a =
        if a<0 then
            super#deljenje (abs(a)*100)
        else
            super#deljenje a
end
\end{lstlisting}
\end{Odgovor}
\begin{Odgovor}{4.14}
\begin{lstlisting}
(* predpostavimo da imamo
narisi_tocko : int*int -> unit
narisi_premico : int*int -> int*int -> unit
narisi_krog : int*int -> int -> unit
*)

let narisi_tocko a b = ();;
let narisi_premico a b = ();;
let narisi_krog a b = ();;

class virtual geo =
object
    method virtual predstavi : string
    method virtual testiraj : float -> float -> bool
    method virtual narisi : unit
end;;

class tocka x y =
object
    inherit geo
    val x = (x:float)
    val y = (y:float)
    method predstavi = "Tocka s koordinatami ("^(string_of_float x)
        ^","^(string_of_float x^").")
    method testiraj tx ty = x=tx && y=ty
    method narisi = narisi_tocko x y
end;;

let t1 = new tocka 1. 2.;;
t1#predstavi;;
t1#testiraj 1. 2.;;
t1#testiraj 1. 2.2;;

class premica k n = (* f(x)=kx+n*)
object
    inherit geo
    val k = k
    val n = n
    method predstavi = "Premica s predpisom f(x)="^(string_of_float k)
        ^"* x + "^(string_of_float n^".")
    method testiraj tx ty = (ty = k *. tx +. n)
    method narisi = narisi_premico (0.,n) (1.,n+.k)
end;;

let p1 = new premica 2. 0.;;
p1#predstavi;;
p1#testiraj 1. 2.2;;
p1#testiraj 1. 2.;;

class krog sx sy r =  (* krog polmera r s srediscem v (sx,sy)*)
object
    inherit geo
    val sx=sx
    val sy=sy
    val r = r
    method predstavi = "Krog s polmerom r="^(string_of_float r)
        ^" ter srediscem v ("^(string_of_float sx)
        ^","^(string_of_float sy)^")."
    method testiraj tx ty = (tx-.sx)**2. +. (ty-.sy)**2. <= r**2.
    method narisi = narisi_krog (sx,sy) r
end

let k1 = new krog 1. 1. 1.;;
k1#predstavi;;
k1#testiraj 1. (-0.);;
k1#testiraj 1. (-0.000001);;
\end{lstlisting}

\end{Odgovor}
\begin{Odgovor}{4.17}

\begin{lstlisting}
type 'a option =
| None
| Some of 'a;;

class ['a] polje n =
object
    val p = ((Array.make n None) : 'a option array)
    method popravi i a = p.(i) <- (Some a)
    method preberi i = p.(i)
end;;

let p1 = new polje 12;;
p1#preberi 2;;
p1#popravi 2 3.14;;
p1#preberi 2;;

class real_polje n =
object
    inherit [float] polje n
    method preberi i = match p.(i) with
    |None -> Some 0.
    |Some x -> Some (float_of_int (int_of_float x))
end
\end{lstlisting}
\end{Odgovor}
\begin{Odgovor}{5.3}

\begin{lstlisting}
module TAgencija =
struct
    type aranzma ={ mutable destinacija:string;
                    mutable tip_namestitve:string;
                    mutable trajanje:int;
                    mutable cena:int}
    let kreiraj dest tip trajanje cena = {destinacija=dest;
                    tip_namestitve=tip;
                    trajanje=trajanje;
                    cena=cena}
    let ogled a = a.destinacija,a.tip_namestitve, a.trajanje,a.cena
    let popravi_dest a nova_dest = a.destinacija <- nova_dest
    let popravi_t_n a nov_t_n = a.tip_namestitve <- nov_t_n
    let popravi_trajanje a novo_trajanje = a.trajanje <- novo_trajanje
    let popravi_ceno a nova_cena = a.cena <- nova_cena
end;;

module type AGENT =
  sig
    type aranzma
    val kreiraj : string -> string -> int -> int -> aranzma
    val ogled : aranzma -> string * string * int * int
    val popravi_dest : aranzma -> string -> unit
    val popravi_t_n : aranzma -> string -> unit
    val popravi_trajanje : aranzma -> int -> unit
    val popravi_ceno : aranzma -> int -> unit
  end;;

module type STRANKA =
  sig
    type aranzma
    val ogled : aranzma -> string * string * int * int
  end;;

module Agent = (TAgencija:AGENT with type aranzma = TAgencija.aranzma);;
module Stranka = (TAgencija:STRANKA with type aranzma = TAgencija.aranzma);;

\end{lstlisting}

\end{Odgovor}
\begin{Odgovor}{5.4}

\textbf{Naloga a)}
\begin{lstlisting}
module Jelka =
struct
    type drevo = {mutable deblo: int ; mutable krosnja : int}
    let ustvari () = {deblo=0; krosnja=0}
    let povejVisino drevo = (drevo.deblo + drevo.krosnja)
    let dolociDeblo drevo v= drevo.deblo <- v
    let dolociKrosnjo drevo v= drevo.krosnja <- v
end
\end{lstlisting}
\textbf{Naloga b)}
\begin{lstlisting}
   (*)
  (***)
 (*****)
(*******)
   (-)
   (-)
\end{lstlisting}
\textbf{Resitev:}
\begin{lstlisting}
    (*deblo vrne string ki predstavlja
    rezino debla v jelki s polmerom krosnje i*)
    let deblo i =
        if i<0 then failwith "Parameter mora biti ne-negativen!" else
        for k = 1 to i do
            print_string " "
        done; print_string "(-)\n"

    (*krosnja vrne j-to vrstico krosnje s polmerom i*)
    let krosnja i j =
        if i<0 || j<1 then failwith "Napacni parametri krosnje!" else
        for k = 1 to i-j+1 do  (*# presledkov*)
            print_string " "
        done; print_string "(";
        for k = 1 to 2*j-1 do  (* # zvezdic *)
            print_string "*"
        done; print_string ")\n"

    let izris d =
        for j = 1 to d.krosnja do
            krosnja (d.krosnja-1) j
        done;
        for j = 1 to d.deblo do
            deblo (d.krosnja-1)
        done
\end{lstlisting}

\end{Odgovor}
\begin{Odgovor}{5.6}

\textbf{Resitev:}
\begin{lstlisting}
module Poljski =
struct
    type t = {mutable sez:int list}
    (*inicializacija: *)
    let kreiraj () = {sez=[]}
    (*vnese stevilo na sklad in ga vrne:*)
    let vnos s i = s.sez <- i::s.sez; i
    (*sesteje vrhnja dva elem iz sklada in vrne rezultat:*)
    let plus s = match s.sez with
        | a::b::rep -> s.sez<- rep; a+b
        | _ -> failwith "Na skladu je premalo stevil!"
    (*odsteje:  *)
    let minus s = match s.sez with
        | a::b::rep -> s.sez<- rep; a-b
        | _ -> failwith "Na skladu je premalo stevil!"
    (*zmnozi:  *)
    let mnozi s = match s.sez with
        | a::b::rep -> s.sez<- rep; a*b
        | _ -> failwith "Na skladu je premalo stevil!"
    (*deli:  *)
    let deli s = match s.sez with
        | a::b::rep -> s.sez<- rep; a/b
        | _ -> failwith "Na skladu je premalo stevil!"
end
\end{lstlisting}
\textbf{Re\v sitev s poljubnim tipom ki podpira osnovne operacije:}
\begin{lstlisting}
module type BASICALGEBRA =
sig
    type aType
    val ( +++ ) : aType -> aType -> aType
    val ( --- ) : aType -> aType -> aType
    val ( *** ) : aType -> aType -> aType
    val ( /// ) : aType -> aType -> aType
end

module MakePoljski = functor (Algebra:BASICALGEBRA) ->
struct
    open Algebra
    type t = {mutable sez : aType list}
    let kreiraj () = {sez=[]}
    let vnos s i = s.sez <- i::s.sez; i
    let plus s = match s.sez with
        | a::b::rep -> s.sez<- rep; a+++b
        | _ -> failwith "Na skladu je premalo stevil!"
    let minus s = match s.sez with
        | a::b::rep -> s.sez<- rep; a---b
        | _ -> failwith "Na skladu je premalo stevil!"
    let mnozi s = match s.sez with
        | a::b::rep -> s.sez<- rep; a***b
        | _ -> failwith "Na skladu je premalo stevil!"
    let deli s = match s.sez with
        | a::b::rep -> s.sez<- rep; a///b
        | _ -> failwith "Na skladu je premalo stevil!"
end

module Reals =
struct
    type aType = float
    let (+++) a b = a+.b
    let (---) a b = a-.b
    let ( *** ) a b = a*.b
    let (///) a b = a/.b
end

module RealPoljski = MakePoljski(Reals);;
open RealPoljski;;
let ss=kreiraj ();;
vnos ss 3.1;;
vnos ss 3.14;;
vnos ss 3.141;;
vnos ss 3.1415;;
plus ss;;
ss;;
\end{lstlisting}

\end{Odgovor}

\end{document}
%</ex1>
%    \end{macrocode}
%
%    \section{A complicated example}
%
%    The following is an (over-complicated) example of the use
%    of package |answers|.
%    It uses some of the refinements described later.
%    \begin{macrocode}
%<*ex2>
\documentclass[12pt,a4paper]{article}
\usepackage{answers}%\usepackage[nosolutionfiles]{answers}
%    \end{macrocode}
%    First an environment which contains problems and numbers them.
%    This is based on a \LaTeX\ theorem, but with a roman body
%    rather than italic.
%    \begin{macrocode}
\newtheorem{Exc}{Exercise}
\newenvironment{Ex}{\begin{Exc}\normalfont}{\end{Exc}}
%    \end{macrocode}
%    Three sorts of solution are written to two different files.
%    File handle |test| will contain the solutions and hints that the
%    students will see; |testtwo| contains the solutions to the
%    problems which they will probably hand in, and so these must be
%    formatted separately.
%    \begin{macrocode}
\Newassociation{solution}{Soln}{test}
\Newassociation{hint}{Hint}{test}
\Newassociation{Solution}{sSol}{testtwo}
%    \end{macrocode}
%    Because we want to mark different types of problem in the master file of
%    problems, we define the following.
%    \begin{macrocode}
\newcommand{\prehint}{~[Hint]}
\newcommand{\presolution}{~[Solution]}
\newcommand{\preSolution}{~[Homework]}
%    \end{macrocode}
%    We provide an extra parameter when we open file handle |test|;
%    this is because we want to write a |\section| command to the
%    solution file.  This is merely an illustration here, but would be
%    more relevant if the solution file were |\include|d.
%    \begin{macrocode}
\newcommand{\Opentesthook}[2]%
   {\Writetofile{#1}{\protect\section{#1: #2}}}
%    \end{macrocode}
%    The default text produced when \LaTeX meets the solution
%    environments is here modified.
%    \begin{macrocode}
\renewcommand{\Solnlabel}[1]{\emph{Solution #1}}
\renewcommand{\Hintlabel}[1]{\emph{Hint #1}}
\renewcommand{\sSollabel}[1]{\emph{Solution to #1}}

\begin{document}
%    \end{macrocode}
%    We open handle |test| as actual file |test1.tex|,
%    \begin{macrocode}
   \Opensolutionfile{test}[ans2]{Solutions}
%    \end{macrocode}
%    and write some text on it.
%    \begin{macrocode}
   \Writetofile{test}{\protect\subsection{Some Solutions}}
%    \end{macrocode}
%    Handle |testtwo| is opened as |testtwo.tex|.
%    \begin{macrocode}
   \Opensolutionfile{testtwo}[ans2x]
   \Writetofile{testtwo}{%
      \protect\subsection{Extra Solutions}}
%    \end{macrocode}
%    Now the problems.
%    \begin{macrocode}
   \section{Exercises}
   \begin{Ex}
      An exercise with a solution.
      \begin{solution}
         This is a solution.
         \relax{}
      \end{solution}
   \end{Ex}
   \begin{Ex}
      An exercise with a hint and a secret solution.
      \begin{hint}
         This is a hint.
      \end{hint}
      \begin{Solution}
         This is a secret solution.
      \end{Solution}
   \end{Ex}
   \begin{Ex}
      An exercise with a hint.
      \begin{hint}
         This is a hint.
      \end{hint}
   \end{Ex}
%    \end{macrocode}
%    We close the two solution files and immediately input their
%    contents.  We could have used |\include| here.
%    \begin{macrocode}
   \Closesolutionfile{test}
   \Readsolutionfile{test}
   \clearpage
   \Closesolutionfile{testtwo}
   \Readsolutionfile{testtwo}
\end{document}
%</ex2>
%    \end{macrocode}
%
%    \section{A further example}
%
%    Here is an application to a situation not originally envisaged,
%    suggested to the author by Martin Osborne.  Here, the exercises and
%    solutions are not numbered; they are \emph{described}.
%    \begin{macrocode}
%<*ex3>
\documentclass[12pt,a4paper]{article}
\usepackage{answers}
\newenvironment{Ex}[1]{\begin{trivlist}\item \emph{#1} %
   \renewcommand{\Currentlabel}{#1}}{\end{trivlist}}
\Newassociation{solution}{Soln}{solutions}

\renewenvironment{Soln}[1]{\begin{trivlist}\item
   Solution to \emph{#1} }{\end{trivlist}}

\begin{document}
\section*{Problems}
   \Opensolutionfile{solutions}[ans3]
   \begin{Ex}{First exercise}
      An exercise with a solution.
      \begin{solution}
         This is a solution.
         \relax{}
      \end{solution}
   \end{Ex}
   \begin{Ex}{Second exercise}
      A second exercise with a solution.
      \begin{solution}
         This is another solution.
      \end{solution}
   \end{Ex}
   \Closesolutionfile{solutions}
\section*{Solutions}
   \Readsolutionfile{solutions}
\end{document}
%</ex3>
%    \end{macrocode}
%
%    \StopEventually{}
%
%
%    \section{Identification}
%
%    This package can only be used with \LaTeXe, so
%    an appropriate message is displayed when another
%    format is used.
%    \begin{macrocode}
%<*answers>
\NeedsTeXFormat{LaTeX2e}
%    \end{macrocode}
%
%
%    Announce the package name and its version:
%    \begin{macrocode}
\ProvidesPackage{answers}
    [2014/08/24 v2.16 Production of solution sheets in LaTeX2e]
%    \end{macrocode}
%
%  \section{Options}
%
%  There is a single option |nosolutionfiles| that
%  switches output off to files
%  and produces the solutions here-and-now.
%
%    \begin{macrocode}
\newif\ifanswerfiles \answerfilestrue
\DeclareOption{nosolutionfiles}{\answerfilesfalse
   \typeout{No answer files being produced}}%
\ProcessOptions

%    \end{macrocode}
%
%
%    As this package now relies heavily on the |verbatim| package, we
%    ensure that that is loaded.
%    \begin{macrocode}
\RequirePackage{verbatim}
%    \end{macrocode}
%
% \begin{macro}{\protected@iwrite}
%   An immediate version of \cs{protected@write}: not available in the
%   kernel but needed for safety.
%    \begin{macrocode}
\long\def\protected@iwrite#1#2#3{%
  \begingroup
    \let\thepage\relax
    #2%
    \let\protect\@unexpandable@protect
    \edef \reserved@a{\immediate\write#1{#3}}%
    \reserved@a
  \endgroup
  \if@nobreak
    \ifvmode
      \nobreak
    \fi
  \fi
}
%    \end{macrocode}
% \end{macro}
%
%    \section{File handling}
%
%    \begin{macro}{\solutionextension}
%    The default extension for solution files is defined here.
%    \begin{macrocode}
\newcommand{\solutionpoint}{.}
\newcommand{\solutionextension}{tex}
%    \end{macrocode}
%    It may be changed with |\renewcommand|.
%    \end{macro}
%
%    \begin{environment}{Filesave}
%    We define an environment |Filesave| with one parameter, the file handle.
%    It is similar to the example of Sch\"opf in the
%    description of |verbatim|.
%    \begin{macrocode}
\newenvironment{Filesave}[1]{%
   \@bsphack
   \def\verbatim@processline{}%
   \Iffileundefined{#1}{}{%
      \Ifopen{#1}{%
         \def\verbatim@processline{%
            \Ifanswerfiles{%
               \immediate\write\@nameuse{#1@file}%
                  {\the\verbatim@line}%
            }{}%
         }%
      }{}%
   }%
   \let\do\@makeother\dospecials
   \catcode`\^^M\active \catcode`\^^I=12\relax
   \verbatim@start
}{\@esphack}
%    \end{macrocode}
%    \end{environment}
%
%    \begin{macro}{\Writetofile}
%    It is also useful to have a command to write material to the file.
%    In this, you need to put |\protect| before any control words in the
%    argument that might expand prematurely and create havoc.
%    \begin{macrocode}
\newcommand{\Writetofile}[2]{%
   \@bsphack
   \Iffileundefined{#1}{}{%
      \Ifopen{#1}{%
         {%
            \let\protect\string
            \Ifanswerfiles{%
               \protected@iwrite{\@nameuse{#1@file}}{}{#2}%
            }{}%
         }%
      }{}%
   }%
   \@esphack
}
%    \end{macrocode}
%    \end{macro}
%
%    \begin{macro}{\Ifopen}
%    We need to check whether or not a file is already open.
%    \begin{macrocode}
\newcommand{\Ifopen}[3]{%
   \csname if#1open\endcsname#2\else#3\fi}%
%    \end{macrocode}
%    \end{macro}
%
%    \begin{macro}{\Iffileundefined}
%    We also need to check whether a file variable has already been defined
%    for a given file handle.
%    \begin{macrocode}
\newcommand{\Iffileundefined}[3]{%
   \csname ifx\expandafter\endcsname
      \csname #1@file\endcsname\relax
      #2\else#3\fi}
%    \end{macrocode}
%    \end{macro}
%    Finally, we need a check as to whether we are outputting answers to a
%    file or not
%    \begin{macrocode}
\newcommand{\Ifanswerfiles}[2]{%
   \ifanswerfiles #1\else #2\fi}
%    \end{macrocode}
%
%
%
%    \section{The file interface}
%
%    \begin{macro}{\Opensolutionfile}
%    Before we can write solutions, we must open the solution file(s).
%    The command to do this takes a single parameter, which should usually be
%    a file name without extension. Thus it should probably be
%    restricted to a string of at most 8 letters for portability.
%    This operation will not truncate any existing open file.
%    However, if the second optional parameter is specified, this
%    determines the actual filename, and the first parameter is then
%    an arbitrary symbolic file handle name.
%    \begin{macrocode}
\def\Opensolutionfile#1{%
   \@ifnextchar[{\define@filename{#1}}%
      {\define@filename{#1}[#1]}}%
\def\define@filename#1[#2]{%
   \global\@namedef{#1@filename}{#2\solutionpoint\solutionextension}%
   \Ifanswerfiles{%
      \typeout{Output from handle #1 going
         to #2.\solutionextension}%
   }{}%
   \Iffileundefined{#1}{%
      \expandafter\newwrite\csname #1@file\endcsname
      \csname newif\expandafter\endcsname
         \csname if#1open\endcsname
      \global\csname #1openfalse\endcsname
      \expandafter\ifx\csname Open#1hook\endcsname\relax
         \global\@namedef{Open#1hook}##1{}%
      \fi
      \expandafter\ifx\csname Close#1hook\endcsname\relax
         \global\@namedef{Close#1hook}##1{}%
      \fi
   }{}%
   \let\Tmp\relax
   \Ifopen{#1}{\typeout{File #1 already open}}{%
      \Ifanswerfiles{%
         \immediate\openout\@nameuse{#1@file}=%
         \@nameuse{#1@filename}%
      }{}%
      \global\csname#1opentrue\endcsname
      \def\Tmp{\@nameuse{Open#1hook}{#1}}%
   }%
   \Tmp
}
%    \end{macrocode}
%    \begin{macro}{\Closesolutionfile}
%    We also have a command to close an already open file.
%    \begin{macrocode}
\def\Closesolutionfile#1{%
   \let\Tmp\relax
   \Iffileundefined{#1}{}{%
      \Ifopen{#1}{%
         \Ifanswerfiles{%
            \immediate\closeout\@nameuse{#1@file}%
         }{}%
         \global\csname #1openfalse\endcsname
         \def\Tmp{\@nameuse{Close#1hook}{#1}}%
      }{}%
   }%
   \Tmp
}
%    \end{macrocode}
%
%    Note that the two file commands each provide a hook which allows them to
%    perform extra tasks.
%    For instance, the opening operation could be made to
%    write extra information to the file by redefining the appropriate hook.
%    The closing operation could if required do an immediate |\input| of the
%    solution file contents. For example,
%    \begin{verbatim}
%        \newcommand{\Openxxxhook}[2]{%
%           \Writetofile{#1}{\protect\section{#2}}%
%        }%
%        \newcommand{\Closexxxhook}[1]{%
%           \Readsolutionfile{#1}%
%        }
%    \end{verbatim}
%    and then
%    \begin{verbatim}
%        \Opensolutionfile{xxx}{Answers to selected problems}
%        ...
%        \Closesolutionfile{xxx}
%    \end{verbatim}
%    The default behaviour is to ignore the one parameter. Note that if you
%    redefine their behaviour, you must remember that the first parameter is
%    always the file handle.
%    \end{macro}
%    \end{macro}
%
%    \begin{macro}{\Readsolutionfile}
%    The operation of reading the file of solutions can be done with the
%    following command.
%    \begin{macrocode}
\def\Readsolutionfile#1{%
   \Ifanswerfiles{%
      \Iffileundefined{#1}{}{%
         \Ifopen{#1}{%
            \typeout{WARNING: attempt to read open file #1}%
         }{%
            \edef\Tmp{%
               \noexpand\InputIfFileExists
                  {\@nameuse{#1@filename}}{}%
               {\noexpand\message{File
                  \@nameuse{#1@filename}%
                     \space not found}}%
            }%
            \Tmp
         }%
      }%
   }{}%
}

%    \end{macrocode}
%    \end{macro}
%
%
%    \section{The solution interface}
%
%    \begin{macro}{\Newassociation}
%    Several solution file handles may have been defined.
%    You are limited only by the
%    number that \TeX\ will make available to you. Each solution environment
%    that is to write to one of these handles
%    must know which handle to write to,
%    and also what extra information to write there, apart from its contents.
%    This is done by setting up an association
%    between the source environment,
%    the destination environment and the file handle.
%    \begin{macrocode}
\newcommand{\Newassociation}[3]{%
   \newsolution{#2}%
   \expandafter\ifx\csname #3opentrue\endcsname\relax
      \expandafter\newif\csname if#3open\endcsname
   \fi
   \newenvironment{#1}{%
      \Ifanswerfiles{%
         \let\Tmp\relax
         \Iffileundefined{#3}{}{%
            \Ifopen{#3}{%
               \immediate\write\@nameuse{#3@file}%
                  {\string\begin{#2}\@nameuse{#2params}}%
            \def\Tmp{\Filesave{#3}}%
            }{}%
         }%
      }{%
         \edef\Tmp{\noexpand\begin{#2}\@nameuse{#2params}}%
      }%
      \csname pre#1\endcsname
      \Tmp
   }%
   {%
      \Ifanswerfiles{%
         \Iffileundefined{#3}{}{%
            \Ifopen{#3}{%
               \endFilesave%
               \immediate\write\@nameuse{#3@file}%
                  {\string\end{#2}}%
               \csname post#1\endcsname
            }{}%
         }%
      }{%
         \end{#2}%
      }%
   }%
}
%    \end{macrocode}
%    \end{macro}
%
%    \begin{macro}{\newsolution}
%    The default destination environment in the solution file is defined to
%    take a single parameter, a reference number inherited from the source
%    environment.  This is set with style |\solutionstyle|, which defaults
%    to |\textbf|.  In addition, solution type |yyy| can have markup added
%    before and after it by defining |\preyyy| and |\postyyy| suitably,
%    eg, a rule across the width of the page or a square.
%    If anything more sophisticated is intended, it is probably better to
%    |\renewenvironment{yyy}| to achieve it.
%    \begin{macro}{\Currentlabel}
%    \begin{macrocode}
\newcommand{\newsolution}[1]{%
   \@ifundefined{#1}{%
      \global\@definecounter{#1}%
      \global\@namedef{#1params}{{\Currentlabel}}%
      \newenvironment{#1}[1]%
      {%
         \refstepcounter{#1}%
         \csname pre#1\endcsname
         \trivlist
         \item[\hskip\itemsep{\@nameuse{#1label}{##1}}]}%
      {\csname post#1\endcsname\endtrivlist}%
      \global\@namedef{#1label}##1{\solutionstyle{##1}}%
   }{\typeout{WARNING: environment #1 already in use}}%
}
\newcommand{\solutionstyle}[1]{\textbf{#1}}
\newcommand{\Currentlabel}{\@currentlabel}
%    \end{macrocode}
%    The format of the label for solution environment |xxx| is governed
%    by the command |\xxxlabel|, which takes one argument by default.
%    The argument is passed to it by the command |\xxxparams|,
%    which expands to |{\Currentlabel}|, a synonym for |{\@currentlabel}|,
%    and this argument is written automatically by the source environment.
%    The label appears in boldface by default.
%    We could easily change the behaviour of this environment by changing
%    these two commands. For example
%    \begin{verbatim}
%        \renewcommand{\xxxlabel}[1]{\emph{Solution to #1}}
%        \renewcommand{\xxxparams}{{\Currentlabel(p.\thepage)}}
%    \end{verbatim}
%    would provide a number and page reference in italic.
%
%    More complicated behaviour could be produced by redefining the |xxx|
%    environment itself to take a different number of parameters. Note
%    however that |\xxxparams| must be redefined to provide those parameters.
%    \end{macro}
%    \end{macro}
%
%    Which brings us to the end of the |answers| package.
%    \begin{macrocode}
%</answers>
%    \end{macrocode}
%
%
%
% \Finale
%
\endinput
%
